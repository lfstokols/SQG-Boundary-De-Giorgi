\documentclass[11pt]{amsart}
\usepackage{amssymb,amsbsy,upref,epsf,MnSymbol}
\usepackage{amsmath,amssymb,amsthm}
\usepackage{graphicx,mathtools,mathrsfs,wrapfig}
\usepackage[margin=1in]{geometry}
\usepackage{fancyhdr}
\usepackage{enumerate}



%-------------------------------------------<commands>--------------------------------------------------------
\newtheorem{theorem}{Theorem}[section]
\newtheorem{proposition}[theorem]{Proposition}
\newtheorem{lemma}[theorem]{Lemma}
\theoremstyle{remark}
\newtheorem*{remark}{Remark}
%%%%%%%%%%%%%%%%%%%%%%%%%%%%%%%%%%%%%%%%%
\newcommand{\R}{\mathbb{R}}
\newcommand{\N}{\mathbb{N}}
\newcommand{\Z}{\mathbb{Z}}
\newcommand{\Q}{\mathbb{Q}}
\newcommand{\C}{\mathbb{C}}
\newcommand{\Prj}{\mathbb{P}}
\newcommand{\F}{\mathbb{F}}
\newcommand{\A}{\mathbb{A}}
\newcommand{\Four}{\mathcal{F}}
\newcommand{\T}{\mathbb{T}}
\newcommand{\E}{\mathbb{E}}
\renewcommand{\L}{\mathcal{L}}
\newcommand{\eps}{\varepsilon}
%%%%%%%%%%%%%%%%%%%%%%%%%%%%%%%%%%%%%%%%%
\newcommand{\floor}[1]{\left\lfloor #1 \right\rfloor}
\newcommand{\ceil}[1]{\left\lceil #1 \right\rceil}
\newcommand{\chevron}[1]{\langle #1 \rangle}
\newcommand{\norm}[1]{\left\lVert#1\right\rVert}
\newcommand{\paren}[1]{\left( #1 \right)}
\newcommand{\bracket}[1]{\left[ #1 \right]}
\newcommand{\abs}[1]{\left\lvert #1 \right\rvert}
%%%%%%%%%%%%%%%%%%%%%%%%%%%%%%%%%%%%%%%%%
\DeclareMathOperator{\id}{id}
\DeclareMathOperator{\convex}{conv}
\DeclareMathOperator{\image}{Im}
\DeclareMathOperator{\im}{Im}
\DeclareMathOperator{\coker}{coker}
\DeclareMathOperator{\supp}{supp}
\DeclareMathOperator{\trace}{tr}
\DeclareMathOperator{\lspan}{span}
\DeclareMathOperator{\conv}{conv} % stands for conv, as in convex hull
\DeclareMathOperator{\Int}{int} % stands for int, as in interior of a set
\DeclareMathOperator{\sign}{sign}
\DeclareMathOperator{\ran}{ran}
\DeclareMathOperator{\rank}{rank}
%\DeclareMathOperator{\dim}{dim}
\newcommand{\dom}{\operatorname{dom}}
\newcommand{\cod}{\operatorname{cod}}
\newcommand{\Hom}{\operatorname{hom}}
\newcommand{\Ob}{\operatorname{Ob}}
\newcommand{\cl}{\operatorname{cl}}
\DeclareMathOperator{\BMO}{BMO}
%%%%%%%%%%%%%%%%%%%%%%%%%%%%%%%%%%%%%%%%%%
\newcommand{\del}{\partial}
\newcommand{\pvec}[2]{\frac{\partial #1}{\partial #2}}
\newcommand{\grad}{\nabla}
\newcommand{\ddt}{\frac{d}{dt}}
\renewcommand{\div}{\operatorname{div}}
\newcommand{\Laplace}{\Delta}
\newcommand{\kinet}{\bracket{\del_t + v\cdot \grad_x}}
\newcommand{\bessel}{\paren{1-\Laplace_v}}
\newcommand{\loc}{\text{loc}}
\newcommand{\ddz}{\frac{d}{dz}}
%%%%%%%%%%%%%%%%%%%%%%%%%%%%%%%%%%%%%%%%%%
\newcommand{\into}{\hookrightarrow}
\newcommand{\onto}{\twoheadrightarrow}
\newcommand{\isom}{\cong}
\newcommand{\rest}{{\upharpoonright}}
\newcommand{\weakly}{\rightharpoonup}
%%%%%%%%%%%%%%%%%%%%%%%%%%%%%%%%%%%%%%%%%%
\newcommand{\ith}{^\mathrm{th}}
\newcommand{\n}{^{-1}}
%%%%%%%%%%%%%%%%%%%%%%%%%%%%%%%%%%%%%%%%%%
\newcommand{\indic}[1]{\chi_{\{#1\}}}
%%%%%%%%%%%%%%%%%%%%%%%%%%%%%%%%%%%%%%%%%%


\newcommand{\eigen}[1]{\eta_{#1}} %my eigenfunctions
\newcommand{\Ctest}{C_c^\infty}
\newcommand{\test}{\mathcal{D}}

\newcounter{step_count}[section]
\newcommand{\step}[1]{\stepcounter{step_count} \smallskip \noindent{\textbf{Step \arabic{step_count}:} #1}}


%-------------------------------------------</commands>--------------------------------------------------------

\newcommand{\draftnum}{1}%-----------------------------UPDATE DRAFT NUM----------------------------------------

\title{SQG Boundary, Draft \draftnum}

\author[Stokols]{Logan F. Stokols} 
\address[L. F. Stokols]{\newline Department of Mathematics, \newline The University of Texas at Austin, Austin, TX 78712, USA}
\email{lstokols@math.utexas.edu}

\date{\today}

%\subjclass[2010]{35H10,35B65,47G20,35Q84} % hypoellptic, regularity, integro-differential, fokker-planck
%\keywords{Fokker-Planck Equation, Fractional Laplacian, H\"older regularity,  De Giorgi method}

%\thanks{\textbf{Acknowledgment.} This work was partially supported by the NSF Grant DMS 1614918. }

\begin{document}




\maketitle \centerline{\date}

We're gonna consider the equation 
\begin{equation}
\del_t \theta + u\cdot \grad \theta + \Lambda \theta = 0,
\\ u = \grad^\perp \Lambda\n \theta.
\end{equation}

Here the operator 
\[ \Lambda := \sqrt{-\Laplace_D} \]
where $\Laplace_D$ is the Laplacian with Dirichlet boundary condition.  

We're going to linearize the equation by fixing $u$ independent of $\theta$.  What property do we want $u$ to have?  For some constant $\kappa$, we'll want
\begin{align*} 
u &= \sum_{j \in \Z} u_j, \\
\norm{\Lambda^{-1/4} u_j}_\infty &\leq \kappa 2^{-j/4}, \\
\norm{\grad u_j}_\infty &\leq \kappa 2^j. 
\end{align*}
The convergence of that sum is in, say, weak $L^2$.  


\section{Lemmas}

\begin{lemma}
If $f$ and $g$ are non-negative functions with disjoint support (i.e. $f(x)g(x) = 0$ for all $x$), then 
\[ \int \Lambda^s f \Lambda^s g \,dx \leq 0. \]
\end{lemma}

This proves, in particular, that $-\int \theta_+ \Lambda \theta_-$ is a positive term (hence dissipational and extraneous) and that $\int \Lambda^{1/2} (\theta-\psi) \Lambda^{1/2} (\theta-\psi)$ breaks down (bilinearly) into the doubly positive, the doubly negative, and the cross term, all of which are positive and hence each of which is positive.  

\begin{proof}
Use the characterization from Caffarelli-Stinga.  There exist non-negative functions $K(x,y)$ and $B(x)$, depending on the parameter $s$, such that
\[ \int \Lambda^s f \Lambda^s g \,dx = \iint [f(x)-f(y)][g(x)-g(y)] K(x,y) \,dxdy + \int f(x) g(x) B(x) \,dx. \]

Since $f$ and $g$ are non-negative and disjoint, the $B$ term vanishes.  Moreover, the product inside the $K$ term becomes
\[ [f(x)-f(y)][g(x)-g(y)] = -f(x)g(y)-f(y)g(x) \leq 0. \]
Since $K$ is non-negative, the result follows.  
\end{proof}

\begin{lemma}
For all functions $f$ in $H_D^1$,
\[ \int \abs{\grad f}^2 = \int \abs{\Lambda f}^2. \]

Moreover, if $f \in H_D^1$ then $\trace(f)=0$.  
\end{lemma}

\begin{proof}
Let $\eigen{i}$ and $\eigen{j}$ be two eigenfunctions of the Dirichlet Laplacian on $\Omega$.  Note that these functions are smooth in the interior of $\Omega$.  Because $\Omega$ has Lipschitz boundary, and because $\eigen{i} \grad \eigen{j}$ is smooth on $\Omega$ and countinuous and bounded on $\overline{\Omega}$ vanishing on the boundary, therefore 
\[ \int_\Omega \div(\eigen{i} \grad \eigen{j}) = \int_{\del \Omega} \eigen{i} \grad \eigen{j}. \]
But $\eigen{i} \grad \eigen{j}$ vanishes on the boundary, so the right hand side vanishes.  Moreover, $\div(\eigen{i} \grad \eigen{j}) = \grad \eigen{i} \cdot \grad \eigen{j} + \eigen{i} \Laplace \eigen{j}$.  Therefore
\[ \int \grad \eigen{i} \cdot \grad \eigen{j} = - \int \eigen{i} \Laplace \eigen{j} = \lambda_k \int \eigen{i} \eigen{j}. \]
Of course, the inner product of two eigenfunctions is 0 unless they are the same eigenfunction, in which case it is 1.  

Consider a function $f = \sum f_k \eigen{k}$ which is an element of $H_D^1$, by which we mean $\sum \lambda_k f_k^2 < \infty$.  Since $\norm{\grad \eigen{k}}_{L^2(\Omega)} = \sqrt{\lambda_k}$, the following sums all converge in $L^2(\Omega)$ and hence the calculation is justified:
\begin{align*}
\int \abs{\grad f}^2 &= \int \paren{\sum_i f_i \grad \eigen{i} } \paren{\sum_j f_j \grad \eigen{j}}
\\ &= \int \sum_{i,j} (f_i f_j) \grad \eigen{i} \cdot \grad \eigen{j}
\\ &= \sum_{i,j} (f_i f_j) \int \grad \eigen{i} \cdot \grad \eigen{j}.
\end{align*}
Since this double-sum vanishes except on the diagonal, we see from [citation] that in fact
\[ \norm{\grad f}_{L^2(\Omega)} = \norm{\Lambda f}_{L^2(\Omega)}. \]

To see that $\trace(f)$ vanishes, note that $f = \sum_{k=0}^\infty f_k \eigen{k}$ and that each finite partial sum for this series satisfies the Dirichlet boundary condition.  Since $\trace$ is a bounded operator on $H^1$, we need only show that this series is Cauchy in $H^1$, in which case its $H^1$ limit will exist and be equal to its $L^2$ limit which will be equal to $f$.  

For each $k$,
\[ \norm{ f_k \eigen{k} }_{H^1} \leq C_\textrm{Poincare} f_k \norm{\grad \eigen{k}}_2 = C f_k \sqrt{\lambda_k}. \]
This sequence is $\ell^2$ summable, since $f \in H_D^1$ by assumption.  Therefore $f$, being an $H^1$ limit of functions with vanishing trace, also has vanishing trace.  
\end{proof}

\begin{lemma}
For any function $f$, and any $0 < s < 1$,
\[ \int \abs{\Lambda^s f}^2 \simeq \int \abs{\paren{-\Laplace}^{s/2} \bar{f}}^2. \]
Here $\bar{f}$ is the extension of $f$ to $\R^2$ and $\paren{-\Laplace}^s$ is defined in the fourier sense.  
\end{lemma}

\begin{proof}
Let $g$ be any Schwarz function in $L^2(\R^2)$, and let $f$ be a function in $H^{s+1}_D$.  Let $E:H^1(\Omega) \to H^1(\R^2)$ be a bounded extension operator, where $H^1$ denotes the classical Sobolev space defined using the gradient.  Define the function
\[ \Phi(z) = \int_{\R^2} \paren{-\Laplace}^{z/2} g E \Lambda^{s-z} f. \]

When $\Re(z) = 0$, then $\norm{\paren{-\Laplace}^{z/2} g}_2 = \norm{g}_2$ and $\norm{\Lambda^{s-z} f}_2 = \norm{\Lambda^s f}_2$.  Hence
\[ \Phi(z) \leq \norm{g}_2 \norm{f}_{H^s_D}. \]

When $\Re(z)=1$, then $\norm{\paren{-\Laplace}^{(z-1)/2} g}_2 = \norm{g}_2$ and 
\[ \norm{\paren{-\Laplace}^{1/2} E\Lambda^{s-z} f}_{L^2(\R^2)} = \norm{\grad E \Lambda^{s-z} f}_{L^2(\R^2)} \leq \norm{E} \norm{\grad \Lambda^{s-z} f}_{L^2(\Omega)}. \]
It remains to ask whether $\Lambda^{s-z} f$ is in $H^1_D$ so that we can apply lemma [citation].  However, this is true based on our assumption $f \in H^{1+s}_D$, since the various powers of $\Lambda$ all commute and form a semigroup.  Ergo
\[ \norm{\grad \Lambda^{s-z} f}_{L^2(\Omega)} = \norm{\Lambda \Lambda^{s-z} f}_2 \leq \norm{\Lambda^s f}_2 \]
and we can bound
\[ \Phi(z) \leq \norm{E} \norm{g}_2 \norm{f}_{H^s_D}. \]

Now we will bound the derivative of $\Phi(z)$.  Specifically, compute the derivative in $z$ of the integrand, for $0<\Re(z)<1$, and hope that it is integrable.  To this end, we rewrite the integrand of $\Phi$ as
\[ \Four\n\paren{ |\xi|^z \hat{g} } E \sum_k \lambda_k^{\frac{s-z}{2}} f_k. \]
The derivative $\ddz$ commutes with linear operators like $\Four\n$ and $E$, so the derivative is
\[ \Four\n\paren{ \ln(|\xi|) |\xi|^z \hat{g} } E \sum_k \lambda_k^{\frac{s-z}{2}} f_k + \Four\n\paren{ |\xi|^z \hat{g} } E \sum_k \frac{-1}{2} \ln(\lambda_k) \lambda_k^{\frac{s-z}{2}} f_k. \]

Since $0<\Re(z)<1$, $\ln(|\xi|)|\xi|$ is bounded as a multiplier operator from Schwarz functions to $L^2$.  Moreover, $\ln(\lambda_k) \lambda_k^{\frac{s-z}{2}} \leq C \lambda_k^{\frac{s-z+\eps}{2}}$ for some $C$ independent of $k$ but dependent on $z$, $\eps$.  Since $f \in H^{1+s}_D$ this sum converges in $L^2$, in fact in $H_D^1$.  This makes our differentiated integrand a sum of two $H^1$ functions with compact support multiplied by two Schwarz functions.  In particular it is integrable, which means we can interchange the integral sign and the derivative $\ddz$ and prove that $\Phi'(z)$ is finite for all $0<\Re(z)<1$. 

This is sufficient now to apply the Hadamard three-lines lemma to our function $\Phi$.  

It follows that for any Schwarz function $g \in L^2(\R^n)$ and $H_D^{s+1}$ function $f$, 
\[ \int_{\R^2} \paren{-\Laplace}^{s/2} g E f = \Phi(s) \leq \norm{g}_{L^2(\R^2)} \norm{f}_{H_D^s}. \]

Since Schwarz functions are dense in $L^2(\R^2)$, this means by density that 
\[ \int \abs{ \paren{-\Laplace}^{s/2} E f }^2 \leq \int \abs{\Lambda^s f}^2 \]
or in other words it means that $E$ is a bounded operator from $H_D^s$ to $H^s$, at least on the subset $H_D^{s+1} \cap H_D^s$.  It remains to extend this bound to the whole space by density.  

We know from [citation] Caffarelli and Stinga that $\test(\Omega)$ is dense in $H_D^s$ for all $0 \leq s < 1$.  In fact, this takes a bit of interpretation, so I ought to illucidate that this is because $H_D^s = H_0^s$ (the latter in the Slobodekij sense) for most $s$ and at $s=1/2$ we get the Lions-Magenes spaces which still has $\test(\Omega)$ dense.  

Surely, right(?), test functions are all inside of $H_D^{1+s}$.  I should meditate on this, but it must be true.  
\end{proof}


\section{De Giorgi Estimates}

First let us derive an energy inequality.  

We know a priori that $\theta \in L^\infty(0,T; L^2(\Omega)) \cap L^2(0,T; H_D^{1/2}(\Omega))$.  Let $\psi: \Omega \to \R^+$ be a non-negative function in $H_D^{1/2}$ non-uniformly, and define $\theta = \theta_+ + \psi - \theta_-$.  Since $\theta - \psi$ is in $H_D^{1/2}$, by the lemma above, both $\theta_+$ and $\theta_-$ are in that space as well.  In particular, our weak solution can eat $\theta_+$.  

We end up with
\[ 0 = \int \theta_+ \bracket{ \ddt + u \cdot \grad + \Lambda } \paren{\theta_+ + \psi - \theta_-} \]
which decomposes into three terms, corresponding to $\theta_+$, $\psi$, and $\theta_-$.  We analyze them one at a time.  

Firstly,
\begin{align*} 
\int \theta_+ \bracket{ \ddt + u \cdot \grad + \Lambda } \theta_+ &= (1/2) \ddt \int \theta_+^2 + (1/2) \int \div u \, \theta_+^2 + \int \abs{\Lambda^{1/2} \theta_+}^2
\\ &= (1/2) \ddt \int \theta_+^2 + \int \abs{\Lambda^{1/2} \theta_+}^2.
\end{align*}

The $\psi$ term produces important error terms:
\begin{align*} 
\int \theta_+ \bracket{ \ddt + u \cdot \grad + \Lambda } \psi &= \ddt \int \theta_+\psi + \int \theta_+ u \cdot \grad \psi + \int \Lambda^{1/2} \theta_+ \Lambda^{1/2} \psi.
\end{align*}

Since $\theta_+ and \theta_-$ have disjoint support, the $\theta_-$ term is nonnegative by lemma [citation]:
\begin{align*} 
\int \theta_+ \bracket{ \ddt + u \cdot \grad + \Lambda } \theta_- &= (1/2) \int \theta_+ \del_t \theta_- + \int \theta_+ u \cdot \theta_- + \int \Lambda^{1/2} \theta_+ \Lambda^{1/2} \theta_-
\\ &= \int \Lambda^{1/2} \theta_+ \Lambda^{1/2} \theta_- \leq 0.
\end{align*}

Put together, we arrive at 
\[ (1/2) \ddt \int \theta_+^2 + \int \abs{\Lambda^{1/2} \theta_+}^2 \leq \abs{\iint \Lambda^{1/2} \theta_+ \Lambda^{1/2} \psi} + \abs{\int \theta_+ u \cdot \grad \psi}. \]

\hrule%----%--------%--------

\[ (1/2) \ddt \int \theta_+^2 + \int u \cdot \grad \frac{\theta_+^2}{2} + \int \theta_+ u \cdot \grad \psi - \int \theta_+ u \cdot \grad \theta_- + \int \theta_+ \Lambda \theta = 0. \]

We break up the $\theta_+ \Lambda \theta$ term into
\begin{align*} 
\int \theta_+ \Lambda \theta &= \int \abs{\Lambda^{1/2} \theta_+} + \int \Lambda^{1/2} \theta_+ \Lambda^{1/2} \psi - \int \Lambda^{1/2} \theta_+ \Lambda^{1/2} \theta_-
\\ &= \int \abs{\Lambda^{1/2} \theta_+}^2 + \iint [\theta_+(x)-\theta_+(y)][\psi(x)-\psi(y)] K(x,y) + \int \theta_+ \psi B - \int  \Lambda^{1/2} \theta_+ \Lambda^{1/2} \theta_-.
\end{align*}
The $\theta_-$ term is non-negative by lemma [citation], and the $B$ term is non-negative since $B \geq 0$, so we have the inequality
\[ (1/2) \ddt \int \theta_+^2 + \int \abs{\Lambda^{1/2} \theta_+}^2 \leq \abs{\iint [\theta_+(x)-\theta_+(y)][\psi(x)-\psi(y)] K(x,y)} + \abs{\int \theta_+ u \cdot \grad \psi}. \]

This integral is symmetric in $x$ and $y$, and the integrand is only nonzero if one of $\theta_+(x)$ and $\theta_+(y)$ is nonzero.  Hence
\[ \iint [\theta_+(x)-\theta_+(y)][\psi(x)-\psi(y)] K(x,y) \leq 2 \indic{\theta_+>0}(x) \abs{[\theta_+(x)-\theta_+(y)][\psi(x)-\psi(y)]} K(x,y). \]
Now we can break up this integral using the Peter-Paul variant of H\"{o}lder's inequality.  
\[ \iint [\theta_+(x)-\theta_+(y)][\psi(x)-\psi(y)] K(x,y) \leq \eps \int \abs{\Lambda^{1/2}\theta_+}^2 + \frac{1}{\eps} \iint \indic{\theta_+>0}(x) [\psi(x)-\psi(y)]^2 K(x,y). \]

It remains to bound the quantity $[\psi(x)-\psi(y)]^2 K(x,y)$.  By Caffarelli-Stinga theorem 2.4 [citation], there is a universal constant $C$ such that
\[ K(x,y) \leq \frac{C}{|x-y|^{3}}. \]
The cutoff $\psi$ is Lipschitz, and it grows at a rate $|x|^\gamma$.  Therefore 
\[ [\psi(x)-\psi(y)]^2 K(x,y) \leq |x-y|^{-1} \wedge |x-y|^{2\gamma-3}. \]
If $3-2\gamma > 2$ then this quantity is integrable.  Thus
\[ \ddt \int \theta_+^2 + \int \abs{\Lambda^{1/2} \theta_+}^2 \lesssim \int \theta_+ u \cdot \grad \psi + \int \indic{\theta_+>0}.\]

%\begin{align*}
%\iint [\theta_+(x)-\theta_+(y)][\psi(x)-\psi(y)] K(x,y) &\leq \eps \iint [\theta_+(x)-\theta_+(y)]^2 K(x,y) + \frac{1}{\eps} \iint_{\theta_+(x) > 0 \textrm{ or } \theta_+(y) > 0} [\psi(x)-\psi(y)]^2 K(x,y)
%\\ &\leq \eps \int \abs{\Lambda^{1/2} \theta_+}^2 + \frac{2}{\eps} \int \indic{\theta_+>0}(x)\int [\psi(x)-\psi(y)]^2 K(x,y)\,dy\,dx.
%\end{align*}

For the drift term, let's say that $u$ is broken down into $u_l$ and $u_h$ (standing for high-pass and low-pass) and that they have the desired properties.  

For the high-pass term, for each $i = 1,2$,
\[ \int (u_l)_i \theta_+ \del_i \psi = \int \Lambda^{-1/4} u_l \Lambda^{1/4} (\theta_+ \grad \psi) \leq \paren{\int \abs{\Lambda^{1/4} \theta_+ \grad \psi}^2}^{1/2} \leq \]

We're looking at $\int g \Lambda^{1/4} \theta_+$ where $\Lambda^{1/4} g = u_h$ and $g \in L^\infty$.  This breaks down as
\[ \iint [g-g^*][\theta_+-\theta_+^*] K_{1/4} + \int g \theta_+ B_{1/4}. \]

For a given parameter $\lambda$, break up into the region where $|x-y|$ is bigger and smaller than $\lambda$.  Considering the bigger part,
\[ \iint_{\geq \lambda} [g-g^*][\theta_+ - \theta_+^*] K_{1/4} \leq 2 (2 \norm{g}_\infty) \iint_{\geq \lambda} |\theta_+ - \theta_+^*| \frac{dxdx^*}{|x-y|^{2+1/4}} \leq 8 \lambda^{-2.25} \norm{g}_\infty \norm{\theta_+}_1. \]

The $B$ part may be a real problem.  The $g$ means it doesn't have a sign, so we actually have to bound it.  Consider this.  
\begin{align*}
\int g \theta_k B_{1/4} &\leq \norm{g}_\infty \int \theta_{k-1} \theta_k B_{1/4}
\\ &\leq \norm{g}_\infty \int \theta_{k-1}^2 B_{1/4}
\\ &\leq \norm{g}_\infty \int \abs{\Lambda^{1/8} \theta_{k-1}}^2.
\end{align*} 

Lastly, we have the near part.  
\begin{align*} 
\iint_{< \lambda} [g-g^*][\theta_+ - \theta_+^*] K_{1/4} &\leq 2\iint_{<\lambda} \chi_+ [g-g^*]^2 |x-y|^{3/4} K_{1/4} + \iint_{<\lambda} [\theta_+-\theta_+^*]^2 |x-y|^{-3/4} K_{1/4} 
\\ &\leq 4 \norm{g}_\infty^2 \int \chi_+ \int_{<\lambda} |x-y|^{-1.5} + \iint_{<\lambda} [\theta_+-\theta_+^*]^2 K_1
\\ &\leq \norm{g}_\infty^2 \lambda^{1/2} \abs{\chi_+} + \int \abs{\Lambda^{1/2} \theta_+}^2
\end{align*}

Taken together,
\[ \int \Lambda^{-1/4} \theta_+ u_h \cdot \grad \psi \leq \frac{1}{2} \int \abs{\Lambda^{1/2} \theta_+ \grad \psi} + \int \chi_+ + \int \theta_+ + \int \theta_+ \grad \psi B_{1/4}. \]


\section{Control on $u$}
Let's assume that our drift term is a sum of $u_j$ for $j \in \Z$ and a constant which is tbd.  Assume that each $u_j$ is an $L^\infty$ function, and that their sum converges to $u$ in $L^2(\Omega)$, and that each $u_j$ specfies the bounds as stated.  First we show that they sum up to $u_h$ and $u_l$ in the ways desired.  Then we show that the properties required are maintained as we zoom.  Then at last we argue that, before any zooming, $u$ really does have this property.  

Firstly, assume that 
\[ u = \lim_{L^2} \sum_{-N}^N u_j, \]
\begin{align*} 
\norm{\Lambda^{-1/4} u_j}_\infty &\leq \kappa 2^{-j/4}, \\
\norm{\grad u_j}_\infty &\leq \kappa 2^j. 
\end{align*}
We then define
\[ u_h = \sum_{j=0}^\infty u_j \]
and 
\[ u_\ell = \sum_{-\infty}^{j=-1} u_j. \]

Since $u_j \in L^\infty$ in particular they are $L^2$ functions which sum in $L^2$.  Remember that only finitely many negative $j$ have $u_j \neq 0$.  The sequence $u_j$ is thus singly infinite and in particular is a Cauchy sequence, so $u_h$ also converges in $L^2$.  Since $\Lambda^{-1/4}$ is a continuous linear operator, it passes to the partial sums and so
\[ \Lambda^{-1/4} u = \lim_{L^2} \sum_{j=0}^\infty \Lambda^{-1/4} u_j. \]
In particular, the sum converges in the sense of distributions, i.e. in $\test(\Omega)'$.  Since test functions are dense in $L^1(\Omega)$, and the partial sums are uniformly bounded in the dual of $L^1(\Omega)$ (namely $L^\infty(\Omega)$), therefore the limit $\Lambda^{-1/4} u_h$ is also bounded in the dual of $L^1(\Omega)$.  
\[ \norm{\Lambda^{1/4} u}_\infty \leq \sum_{j=0}^\infty \norm{\Lambda^{-1/4} u_j}_\infty \leq \kappa \frac{1}{1-2^{-1/4}}. \]








\end{document}