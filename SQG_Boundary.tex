\documentclass[11pt]{amsart}
\usepackage{amssymb,amsbsy,upref,epsf,MnSymbol}
\usepackage{amsmath,amssymb,amsthm}
\usepackage{graphicx,mathtools,mathrsfs,wrapfig}
\usepackage[margin=1in]{geometry}
\usepackage{fancyhdr}
\usepackage{enumerate}
%\usepackage{enumitem}



%-------------------------------------------<commands>--------------------------------------------------------
\newtheorem{theorem}{Theorem}[section]
\newtheorem{proposition}[theorem]{Proposition}
\newtheorem{lemma}[theorem]{Lemma}
\theoremstyle{remark}
\newtheorem*{remark}{Remark}
\theoremstyle{definition}
\newtheorem{definition}{Definition}
%%%%%%%%%%%%%%%%%%%%%%%%%%%%%%%%%%%%%%%%%
\newcommand{\R}{\mathbb{R}}
\newcommand{\N}{\mathbb{N}}
\newcommand{\Z}{\mathbb{Z}}
\newcommand{\Q}{\mathbb{Q}}
\newcommand{\C}{\mathbb{C}}
\newcommand{\Prj}{\mathbb{P}}
\newcommand{\F}{\mathbb{F}}
\newcommand{\A}{\mathbb{A}}
\newcommand{\Four}{\mathcal{F}}
\newcommand{\T}{\mathbb{T}}
\newcommand{\E}{\mathcal{E}}
\renewcommand{\L}{\mathcal{L}}
\newcommand{\eps}{\varepsilon}
%%%%%%%%%%%%%%%%%%%%%%%%%%%%%%%%%%%%%%%%%
\newcommand{\floor}[1]{\left\lfloor #1 \right\rfloor}
\newcommand{\ceil}[1]{\left\lceil #1 \right\rceil}
\newcommand{\chevron}[1]{\langle #1 \rangle}
\newcommand{\norm}[1]{\left\lVert#1\right\rVert}
\newcommand{\paren}[1]{\left( #1 \right)}
\newcommand{\bracket}[1]{\left[ #1 \right]}
\newcommand{\abs}[1]{\left\lvert #1 \right\rvert}
%%%%%%%%%%%%%%%%%%%%%%%%%%%%%%%%%%%%%%%%%
\DeclareMathOperator{\id}{id}
\DeclareMathOperator{\convex}{conv}
\DeclareMathOperator{\image}{Im}
\DeclareMathOperator{\im}{Im}
\DeclareMathOperator{\coker}{coker}
\DeclareMathOperator{\supp}{supp}
\DeclareMathOperator{\trace}{tr}
\DeclareMathOperator{\lspan}{span}
\DeclareMathOperator{\conv}{conv} % stands for conv, as in convex hull
\DeclareMathOperator{\Int}{int} % stands for int, as in interior of a set
\DeclareMathOperator{\sign}{sign}
\DeclareMathOperator{\ran}{ran}
\DeclareMathOperator{\rank}{rank}
%\DeclareMathOperator{\dim}{dim}
\newcommand{\dom}{\operatorname{dom}}
\newcommand{\cod}{\operatorname{cod}}
\newcommand{\Hom}{\operatorname{hom}}
\newcommand{\Ob}{\operatorname{Ob}}
\newcommand{\cl}{\operatorname{cl}}
\DeclareMathOperator{\BMO}{BMO}
%%%%%%%%%%%%%%%%%%%%%%%%%%%%%%%%%%%%%%%%%%
\newcommand{\del}{\partial}
\newcommand{\pvec}[2]{\frac{\partial #1}{\partial #2}}
\newcommand{\grad}{\nabla}
\newcommand{\ddt}{\frac{d}{dt}}
\renewcommand{\div}{\operatorname{div}}
\newcommand{\Laplace}{\Delta}
\newcommand{\kinet}{\bracket{\del_t + v\cdot \grad_x}}
\newcommand{\bessel}{\paren{1-\Laplace_v}}
\newcommand{\loc}{\text{loc}}
\newcommand{\Lip}{\text{Lip}}
%\newcommand{\BMO}{\text{BMO}}
\renewcommand{\Re}{\operatorname{Re}}
\newcommand{\ddz}{\frac{d}{dz}}
%%%%%%%%%%%%%%%%%%%%%%%%%%%%%%%%%%%%%%%%%%
\newcommand{\into}{\hookrightarrow}
\newcommand{\onto}{\twoheadrightarrow}
\newcommand{\isom}{\cong}
\newcommand{\rest}{{\upharpoonright}}
\newcommand{\weakly}{\rightharpoonup}
%%%%%%%%%%%%%%%%%%%%%%%%%%%%%%%%%%%%%%%%%%
\newcommand{\ith}{^\mathrm{th}}
\newcommand{\n}{^{-1}}
\newcommand{\half}{\frac{1}{2}}
%%%%%%%%%%%%%%%%%%%%%%%%%%%%%%%%%%%%%%%%%%
\newcommand{\indic}[1]{\chi_{\{#1\}}}
%%%%%%%%%%%%%%%%%%%%%%%%%%%%%%%%%%%%%%%%%%


\newcommand{\eigen}[1]{\eta_{#1}} %my eigenfunctions
\newcommand{\Ctest}{C_c^\infty}
\newcommand{\test}{\mathcal{D}}

\newcommand{\ulow}{u_l}
\newcommand{\uhigh}{u_h}
\newcommand{\ulowth}[1]{\ulow^{#1}}
\newcommand{\uhighth}[1]{\uhigh^{#1}}

\newcommand{\HD}{\mathcal{H}}
\newcommand{\HDint}[2]{\int \abs{\Lambda^{#1} #2}^2}
\newcommand{\HDinth}[1]{\HDint{1/2}{#1}}

\newcommand{\Cgamma}{C_g}
\newcommand{\Comega}{C_\Omega}

\newcommand{\Rom}[1]{\MakeUppercase{\romannumeral #1}}

\newcounter{step_count}[section]
\newcommand{\step}[1]{\stepcounter{step_count} \smallskip \noindent{\textbf{Step \arabic{step_count}:} #1}}


%-------------------------------------------</commands>--------------------------------------------------------

%\newcommand{\draftnum}{1}%-----------------------------UPDATE DRAFT NUM----------------------------------------

\title[Boundary Regularity for SQG]{Holder regularity up to the boundary for critical SQG on bounded domains, \today}

\author[Stokols]{Logan F. Stokols} 
\address[L. F. Stokols]{\newline Department of Mathematics, \newline The University of Texas at Austin, Austin, TX 78712, USA}
\email{lstokols@math.utexas.edu}

\author[Vasseur]{Alexis F. Vasseur}
\address[A. F. Vasseur]{\newline Department of Mathematics, \newline The University of Texas at Austin, Austin, TX 78712, USA}
\email{vasseur@math.utexas.edu}

\date{\today}

\subjclass[2010]{35Q35,35Q86} % hypoellptic, regularity, integro-differential, fokker-planck
%Logan: think about keyword for Japanese thing
\keywords{SQG, Bounded domain, Besov spaces on bounded domains, H\"older regularity,  De Giorgi method}

%\thanks{\textbf{Acknowledgment.} This work was partially supported by the NSF Grant DMS 1614918. }

\begin{document}

\begin{abstract}
Logan: write this
\end{abstract}


\maketitle \centerline{\date}

\tableofcontents

The surface quasigeostrophic equation (SQG), proposed by [citation Constantin, Majda, Tabak], is interesting both because of its connection to meteorology and because of its similarities with the incompressible Navier-Stokes equations.  See \cite{Pe} for a detailed review.  The question of global well-posedness has been answered by \cite{CaVa.sqg} and by [citation Kiselev, Nazarov, Volberg].  Two important classes of techniques used in their study is the De Giorgi style techniques utilized by \cite{CaVa.sqg} for SQG and by \cite{NoVa.qg} for the Quasigeostrophic system of equations, and the nonlinear maximum principle techniques utilized by \cite{CoVi} and \cite{CoTaVi}.  

%originates in the study of weather patterns.  It is also important because it is similar to the 2D Navier-Stokes.  The case $\nu > 0$ is called the dissipative SQG, the case $s =1/2$ is called the critical SQG, and $s<1/2$ and $1/2<s<1$ are called super-critical and sub-critical respectively.  Regularity was obtained by \cite{CaVa} and later re-proved and improved upon in [citations].  

In this paper, we consider critical SQG on a bounded domain.  There are several distinct ways to define SQG on a bounded domain, which generally correspond to alternative definitions of the half-Laplacian.  Kriventsov [citation, Dennis Kriventsov] considered a two-phase problem which satisfied SQG only in a region, and was able to prove H\"{o}lder regularity in the time-independent case.  This problem, intended to model air currents over a region containing both land and water, contains a half-Laplacian and a Riesz transform defined, not spectrally, but in terms of extension.  In [citation, Matt and Vasseur], the authors model the full 3D inviscit Qasigeostrophic system in an impermeable cylinder and obtain boundary conditions which, when passed into the SQG submodel, result in a nonlocal operator which is not well understood.  

We will consider the following model, which was introduced by Constantin and Ignatova in \cite{CoIg.fraclap} and \cite{CoIg.sqg}:

For $\Omega$ a bounded domain in $\R^2$ with $C^{2,\beta}$ boundary for some $\beta \in (0,1)$, consider the Laplacian with homogeneous Dirichlet boundary conditions $(-\Laplace_D)$.  If $(\eigen{k})_{k \in \N}$ is a family of eigenfunctions of $-\Laplace_D$ with corresponding eigenvalues $\lambda_k$ listed in increasing order, define
\[ \Lambda f := \sum_{k=0}^\infty \sqrt{\lambda_k} \chevron{f,\eigen{k}}_{L^2(\Omega)} \eigen{k}. \]
The critical SQG problem on $\Omega$ with initial data $\theta_0 \in L^2(\Omega)$ is
\begin{equation} \label{eq:main nonlinear} \begin{cases}
\del_t \theta + u\cdot \grad \theta + \Lambda \theta = 0 & (0,T) \times \Omega,
\\ u = \grad^\perp \Lambda\n \theta & [0,T] \times \Omega,
\\ \theta = \theta_0 & \{0\} \times \Omega.
\end{cases} \end{equation}

Existence of weak solutions for this problem is proven in \cite{CoIg.fraclap}, and interior regularity is proven in \cite{CoIg.fraclap}.  The method of proof for interior regularity is by nonlinear maximum principles, formulas which refine the pointwise Cordoba-Cordoba inequality and allow them to control the dissipation of solutions near spatial maxima.  However, the bounds blow up near the boundary and so the method seems inapplicable to the question of global regularity.  

Our main result will be to show that $\theta$ is globally H\"{o}lder continuous.  In \cite{CoIg.sqg} Remark 1, the question of global regularity was proposed, citing both $C^\alpha(\bar{\Omega})$ bounds and the $C^\alpha(\bar{\Omega}) \implies C^\infty(\bar{\Omega})$ implication as open problems.  To quote Remark 1, these questions are unsolved ``due to the fact that the commutator between normal derivatives and the fractional Dirichlet Laplacian is not controlled uniformly up to the boundary.''  

\begin{theorem} \label{thm:main continuity}
Let $\theta$ be a smooth solution to \eqref{eq:main nonlinear} with initial data $\theta_0 \in L^2(\Omega)$ on a bounded open set $\Omega \subseteq \R^2$ with $C^{2,\beta}$ boundary, $\beta \in (0,1)$, and on a time interval $[0,T]$.  

Then for any $t \in (0,T)$, $\theta$ is H\"{o}lder continuous uniformly on $(t,T)\times\bar{\Omega}$.  

More precisely, there exists $\alpha \in (0,1)$ depending only on $\Omega$ and $\norm{\theta_0}_{L^2(\Omega)}$, and some constant $C$ depending only on $\Omega$ and $t$, such that
\[ \norm{\theta}_{C^\alpha((t,T)\times\Omega)} \leq C \norm{\theta_0}_{L^2(\Omega)}. \]
\end{theorem}

Our method of proof will be the De Giorgi method.  Introduced by De Giorgi in \cite{DG}, the method was applied to the SQG problem in \cite{CaVa.sqg} and to the 3D Quasigeostrophic problem in \cite{NoVa.qg}.  The underlying idea is to forget the dependence of the velocity $u$ on $\theta$ to obtain a linear equation, and prove alternating regularity results for $\theta$ and $u$ independently.  We first show that $\theta$ is in $L^\infty$ regardless of the value of $u$ (Section~\ref{sec:Linfty}).  We use this $L^\infty$ bound on $\theta$ to control $u$ (Section~\ref{sec:littlewood paley}).  We prove a decrease-in-oscillation lemma for fractional diffusion equations with ``small'' drift (Section~\ref{sec:harnack}).  By applying this oscillation lemma, then scaling our equation, and then applying the oscillation lemma again iteratively, we can show that $\theta$ is H\"{o}lder continuous (Section~\ref{sec:holder}.  

In [citation Stinga Torrea] and \cite{CaSt} the extension representation of [citation Caff Silvestre] is applied to the Dirichlet fractional Laplacian, and in \cite{CaSt} this representation is used to derive a singular integral representation for $\Lambda^s$.  Previous applications of the De Giorgi method generally make extensive use of either an extension representation (c.f. \cite{CaVa.sqg}) or a singular-integral representation (c.f. \cite{NoVa.qg}), so this theory is pivotal in translating the existing non-local De Giorgi techniques to the problem at hand.  In the present paper, we lean heavily on the singular integral representation (see Section~\ref{sec:lemmas}).  

The adaptation of Fourier analysis and Littlewood-Paley theory to Schrodinger operators has is a well-studied subject [citations].  As an application of this theory, \cite{IMTs} and [citation, Dong et al] have considered operators defined on open subsets of $\R^n$, which includes as a special case the operator $-Laplace_D$ (a Schrodinger operator with zero potential).  In particular, the recent preprint \cite{IMTb} of Iwabuchi, Matsuyama, and Taniguchi derives many important results, including the Bernstein inequalities, for Besov spaces adapted to the operator $-\Laplace_D$ on bounded open subsets of $\R^n$ with smooth boundary.  This theory turns out to greatly improve our understanding of the Riesz transform $\grad \Lambda^{-1}$ on bounded domains.  

In order to apply the De Giorgi theory, we must find some space in which the velocity $u = \grad \Lambda^{-1} \theta$ is bounded, and this bound must be scaling invariant.  Though $L^\infty$ is the obvious choice, since solutions to SQG are easily shown to be $L^\infty$ bounded under minimal assumptions, the Riesz transform is not bounded from $L^\infty$ to $L^\infty$.  The usual solution is to consider $\BMO$ (as in \cite{CaVa.sqg} and \cite{NoVa.qg}), but in the case of bounded domains the Riesz transform is not known to be bounded in this space either.  

Using the results of \cite{IMTb}, we will be able to show that the Riesz transform of an $L^\infty$ function whose Fourier decomposition $f = \sum f_k \eigen{k}$ is supported on high frequencies $k > N$ will be bounded in the weak sobolev space $W^{-1/4,\infty}$, and the Riesz transform of an $L^\infty$ function whose Fourier decomposition is supported on low frequencies $k<N$ will have bounded Lipschitz constant.  The cutoff $N$ for dividing high frequencies from low frequencies must depend however on the size of the domain $\Omega$.  In the case of $\R^2$, where $\grad$ and $\Lambda^{-1}$ commute, this is equivalent to the observation that the Riesz transform is bounded from $L^\infty$ to the Besov space $B_{\infty,\infty}^0$.  In the case of bounded domains, the argument must be more subtle.  We must decompose $\theta$ into its Littlewood-Paley projections, individually bound the Riesz transform of each projection in multiple spaces, and then recombine these infinitely-many functions into a low-frequency collection and a high-frequency collection depending on the scale of oscillation we are trying to detect.  

We make this notion precise with the following definition.  

\begin{definition}[Calibrated sequence] \label{def:calibrated}
Let $\Omega\subseteq \R^2$ be any bounded open set and $0<T\in\R$.  We call a function $u\in L^2([0,T]\times\Omega)$ \textbf{calibrated} if it can be decomposed as the sum of a calibrated sequence
\[ u = \sum_{j \in \Z} u_j \]
with each $u_j \in L^2([0,T]\times\Omega)$ and the infinite sum converging in the sense of $L^2$.  

We call a sequence $(u_j)_{j\in\Z}$ \textbf{calibrated} for a constant $\kappa$ and a center $N$ if each term of the sequence satisfies the following bounds.  

\begin{align*}
\norm{u_j}_{L^\infty([0,T]\times\Omega)} &\leq \kappa, \\
\norm{\grad u_j}_{L^\infty([0,T]\times\Omega)} &\leq 2^{j} 2^{-N} \kappa, \\
\norm{\Lambda^{-1/4} u_j}_{L^\infty([0,T]\times\Omega)} &\leq 2^{-j/4} 2^{N/4} \kappa.  
\end{align*} 

\end{definition}

In Section \ref{sec:littlewood paley} we will show that $u$ is calibrated and in Section \ref{sec:holder} we will show that it remains calibrated at all scales (specifically, with fixed constant $\kappa$ but not always with the same center $N$).  Therefore we will consider the linear equation
\begin{equation} \label{eq:main linear} \begin{cases}
\del_t \theta + u \cdot \grad \theta + \Lambda \theta = 0, & [-T,0]\times\Omega \\
\div u = 0 & [-T,0]\times\Omega.
\end{cases} \end{equation}
In Section \ref{sec:Linfty} we show that solutions to \eqref{eq:main linear}, with minimally regular velocity and $L^2$ initial data, become $L^\infty$ instantly, and in Sections \ref{sec:de giorgi} and \ref{sec:harnack} we will show that solutions to \eqref{eq:main linear} with calibrated velocity have decreasing oscillation between scales.  

Though it is beyond the scope of the present paper, we believe that global regularity of solutions to \eqref{eq:main nonlinear} can be proven once H\"{o}lder regularity is known.  In appendix 2 of \cite{CoIg.sqg}, local-in-time existence of solutions to \eqref{eq:main nonlinear} with $\theta \in L^\infty(0,t; L^2(\Omega) \cap H_0^1(\Omega) \cap L^4(\Omega))$ and $u \in L^\infty(0,t; L^4(\Omega))$ is proven for initial data sufficiently smooth.  Therefore, going forward, we can assume this much regularity on our solutions.  

\textbf{Notation.}
If $f = \sum_k f_k \eigen{k}$ then
\begin{align*} 
\norm{f}_{\HD^s} &:= \paren{\sum_k \lambda_k^{s} f_k^2}^{1/2} 
\\ &= \HDint{s}{f}. 
\end{align*}
We suppress the dependence on $\Omega$, though in fact the $\lambda_k$ and $\Lambda$ are defined in terms of the domain $\Omega$.  The relevant domain will be clear from context.  This is in fact a norm, not a seminorm, since $\norm{f}_{L^2(\Omega)} \leq \lambda_0^{s/2} \norm{f}_{\HD^s}$.  

Recall the notation
\begin{align*}
\bracket{f}_\alpha &:= \sup_{x,y \in \Omega, x \neq y} \frac{|f(x)-f(y)|}{|x-y|^\alpha},  &\alpha \in (0,1], \\
\norm{f}_{C^\alpha} &:= \norm{f}_\infty + \bracket{f}_\alpha, & \alpha \in (0,1], \\
\norm{f}_{C^{k,\alpha}} &:= \sum_{n=0}^k \norm{D^n f}_\infty + \bracket{D^k f}_\alpha, & \alpha \in (0,1], k \in \N.
\end{align*}

We will use the notation $(x)_+ := \max(0,x)$.  When the parentheses are ommited, the subscript~$+$ is merely a label.  
Throughout this paper, if an integral sign is written $\int$ without a specified domain, the domain is implied to be $\Omega$, with $\Omega$ defined in context.  
For any vector $v = (v_1,v_2)$, by $v^\perp$ we mean $(-v_2,v_1)$, and by $\grad^\perp$ we mean $(-\del_y, \del_x)$.  

\section{Properties of $\Lambda$} \label{sec:lemmas}

We begin by recounting the result of \cite{CaSt} which gives us a singular integral representation of the $\HD^s$ norm.  
\begin{proposition}[Caffarelli-Stinga Representation] \label{thm:Caff Stinga representation}
Let $s \in (0,1)$ and $f,g \in \HD^{s}$ on a bounded $C^{2,\alpha}$ domain $\Omega \subseteq \R^2$.  Then
\[ \int_\Omega \Lambda^s f \Lambda^s g \,dx = \iint_{\Omega^2} [f(x)-f(y)][g(x)-g(y)] K_{2s}(x,y) \,dxdy + \int_{\Omega} f(x) g(x) B_{2s}(x) \,dx \]
for kernels $K_{2s}$ and $B_{2s}$ which depend on the parameter $s$ and the domain $\Omega$.  

Moreover, these kernels are bounded
\[ 0 \leq K_{2s}(x,y) \leq \frac{C(\Omega,s)}{|x-y|^{2+2s}} \]
for all $x \neq y \in \Omega$ and
\[ 0 \leq B_{2s}(x) \]
for all $x \in \Omega$.  

Finally, for any $s,t \in (0,2)$ there exists a constant $c = c(s,t,\Omega)$ such that for all $x\neq y \in \Omega$
\begin{equation} \label{relationship Kt and Ks} K_{t}(x,y) \leq |x-y|^{s-t} K_{s}(x,y). \end{equation}
\end{proposition}

\begin{proof}
See \cite{CaSt} Theorems 2.3 and 2.4.  

Theorem 2.4 in \cite{CaSt} does not explicitly state the result \eqref{relationship Kt and Ks}.  However, it does state that for each kernel $K_s$ there exists a constant $c_s$ dependent on $s$ and $\Omega$ such that 
\[ \frac{1}{c_s} |x-y|^{2+s} K_s(x,y) \leq \min\paren{1,\frac{\eigen{0}(x)\eigen{0}(y)}{|x-y|^2}} \leq c_s |x-y|^{2+s} K_s(x,y). \]
Since the middle term does not depend on $s$, we can say that
\[ |x-y|^{2+t} K_t(x,y) \leq c_t c_s |x-y|^{2+s} K_s(x,y) \]
from which \eqref{relationship Kt and Ks} follows.  
\end{proof}

From the explicit formulae given in \cite{CaSt}, we see that $K_{2s}$ is roughly equal to the standard kernel for the $\R^2$ fractional Laplacian $(-\Laplace)^s$ when both $x$ and $y$ are in the interior of $\Omega$ or when $x$ and $y$ are extremely close together, but decays to zero when one point is in the interior and the other is near the boundary.  The kernel $B_{2s}$ is well-behaved in the interior but has a singularity at the boundary $\del \Omega$.  This justifies our thinking of the $K_{2s}$ term as the interior term and $B_{2s}$ as a boundary term.  

When comparing the computations in this paper to corresponding computations on $\R^2$, it is convenient to say that the interior term behaves roughly the same as in the unbounded case, while the boundary term behaves roughly like a lower order term (in the sense that it is easily localized).  

Many useful results can be derived from Caffarelli-Stinga representation formula, including the following:

\begin{lemma} \label{thm:Lambda stuff}
Let $\Omega \subseteq \R^2$ be a bounded open set with $C^{2,\beta}$ boundary.  

\begin{enumerate}[(a)]
\item \label{thm:disjoint} Let $s \in (0,1)$.  If $f$ and $g$ are non-negative functions with disjoint support (i.e. $f(x)g(x) = 0$ for all $x$), then 
\[ \int \Lambda^s f \Lambda^s g \,dx \leq 0. \]

\item \label{thm:product rule} Let $s \in (0,1)$.  If $g \in C^{0,1}(\Omega)$ then
\[ \norm{fg}_{\HD^s} \leq 2 \norm{g}_\infty \norm{f}_{\HD^s} + 2 \norm{f}_2 \sup_y \int \frac{|g(x)-g(y)|^2}{|x-y|^{2+2s}} \,dx. \]

\item \label{thm:extra product rule} Let $s \in (0,1)$.  If $g \in C^{0,1}(\Omega)$ then for some constant $C=C(s)$
\[  \norm{fg}_{\HD^s} \leq C \norm{g}_{C^{0,1}(\Omega)} \paren{\norm{f}_2 + \norm{f}_{\HD^s}}. \]

\item \label{thm:L1 of Lambda bounded} Let $s\in(0,1/2)$.  Let $g$ an $L^\infty(\Omega)$ function and $f \in \HD^{2s}$ be non-negative with compact support.  Let there be a constant $\Comega$ such that
\begin{equation} \label{K bounded between orders} K_s(x,y) \leq \Comega |x-y|^{3s} K_{4s}(x,y). \end{equation}

Then
\[ \int \Lambda^{s/2} g \Lambda^{s/2} f \leq C \norm{g}_\infty |\supp(f)|^{1/2} \paren{ \norm{f}_2 + \norm{f}_{\HD^{2s}}}. \]

\item \label{thm:L1 of Lambda1/4 bounded} Let $g$ an $L^\infty(\Omega)$ function and $f \in \HD^{1/2}$ be non-negative with compact support.  Let there be a constant $\Comega$ such that
\[ K_{1/4}(x,y) \leq \Comega |x-y|^{3/4} K_{1}(x,y). \]

Then
\[ \int g \Lambda^{1/4} f \leq C \norm{g}_\infty |\supp(f)|^{1/2} \paren{ \norm{f}_2 + \norm{f}_{\HD^{1/2}}}. \]

\end{enumerate}
\end{lemma}

\begin{proof}
We prove these corollaries one at a time.  

\textbf{Proof of \eqref{thm:disjoint}:}
From Proposition \ref{thm:Caff Stinga representation}
\[ \int \Lambda^s f \Lambda^s g \,dx = \iint [f(x)-f(y)][g(x)-g(y)] K(x,y) \,dxdy + \int f(x) g(x) B(x) \,dx. \]

Since $f$ and $g$ are non-negative and disjoint, the $B$ term vanishes.  Moreover, the product inside the $K$ term becomes
\[ [f(x)-f(y)][g(x)-g(y)] = -f(x)g(y)-f(y)g(x) \leq 0. \]
Since $K$ is non-negative, the result follows.  

\textbf{Proof of \eqref{thm:product rule}:}
From Proposition \ref{thm:Caff Stinga representation}
\[ \int |\Lambda^s (fg)|^2 = \iint \paren{g(x)[f(x)-f(y)] + f(y)[g(x)-g(x)]}^2 K + \int f^2 g^2 B \]
\[ \leq 2 \norm{g}_\infty^2 \norm{f}_{\HD^s}^2 + 2 \int f(y)^2 \int \frac{|g(x)-g(y)|^2}{|x-y|^{2+2s}}. \]

\textbf{Proof of \eqref{thm:extra product rule}:}
This follows immediately from \eqref{thm:product rule}.  

\textbf{Proof of \eqref{thm:L1 of Lambda bounded}:}
From Proposition \ref{thm:Caff Stinga representation} we can decompose
\[ \int \Lambda^{s/2} g \Lambda^{s/2} f = \Rom{1}_< + \Rom{1}_\geq + \Rom{2} \]
where
\begin{align*} 
\Rom{1}_< &:= \iint_{|x-y| < 1} [g(x)-g(y)][f(x)-f(y)] K_s, \\
\Rom{1}_\geq &:= \iint_{|x-y|\geq 1} [g(x)-g(y)][f(x)-f(y)] K_s, \\
\Rom{2} &:= \int f g B_s. 
\end{align*}

First we estimate $\Rom{1}_<$.  From \eqref{K bounded between orders} and from the symmetry of the integrand and the fact that $[f(x)-f(y)]$ vanishes unless at least one of $f(x)$ or $f(y)$ is non-zero,
\[ \abs{\Rom{1}_<} \leq 2 \iint_{|x-y| < 1} \indic{f > 0}(x) \abs{g(x)-g(y)} \cdot \abs{f(x)-f(y)} \cdot |x-y|^{3s} K_{4s}. \]
We can break this up by Holder's inequality
\[ \abs{\Rom{1}_<} \leq 2 \paren{\iint_{|x-y| < 1} \indic{f > 0}(x) [g(x)-g(y)]^2 |x-y|^{6s} K_{4s} }^{1/2} \paren{\iint [f(x)-f(y)]^2 K_{4s} }^{1/2}. \]
The kernel $|x-y|^{6s} K_{4s} \indic{|x-y| < 1}$ is integrable in $y$ for $x$ fixed.  Therefore
\begin{equation} \label{K near of L1 less than L2 stuff} \abs{\Rom{1}_<} \leq 2 \paren{ 2\norm{g}_\infty^2 \int C \indic{f > 0}(x) \,dx }^{1/2} \paren{ \norm{f}_{\HD^{2s}}^2}^{1/2}. \end{equation}

For the term $\Rom{1}_\geq$, by the symmetry of the integrand we have
\[ \abs{\Rom{1}_\geq} \leq  2\norm{g}_\infty 2\int |f(x)| \int_{|x-y|\geq 1} K_s(x,y)dy \,dx. \]
Since $K_s \indic{|x-y|\geq 1}$ is integrable in $y$ for $x$ fixed,
\begin{equation} \label{K far of L1 less than L2 stuff}
\abs{\Rom{1}_\geq} \leq C \norm{g}_\infty \norm{f}_1.  
\end{equation}

For the boundary term \Rom{2}, 
\[ \abs{\Rom{2}} \leq \norm{g}_\infty \int\indic{f>0} f B_s. \]
Since $f \geq 0$, $[f(x)-f(y)][\indic{f > 0}(x) - \indic{f > 0}(y)] \geq 0$.  Therefore
\[ \int \indic{f > 0} f B_s \leq \int \Lambda^{s/2} \indic{f>0} \Lambda^{s/2} f = \int \indic{f>0} \Lambda^{s} f. \]
Applying H\"{o}lder's inequality, we arrive at
\[ \abs{\Rom{2}} \leq \norm{g}_\infty |\supp(f)|^{1/2} \norm{f}_{\HD^{s}}. \]
This combined with \eqref{K near of L1 less than L2 stuff} and \eqref{K far of L1 less than L2 stuff} gives us 
\[ \int \Lambda^{s/2} g \Lambda^{s/2} f \leq C \norm{g}_\infty \paren{|\supp(f)|^{1/2} \norm{f}_{\HD^{2s}} + \norm{f}_1 + |\supp(f)|^{1/2} \norm{f}_{\HD^s}}. \]

The lemma follows since $\norm{f}_1 \leq |\supp(f)|^{1/2} \norm{f}_2$ and since $\norm{f}_{\HD^s} \leq \norm{f}_{L^2} + \norm{f}_{\HD^{2s}}$.  

\textbf{Proof of \eqref{thm:L1 of Lambda1/4 bounded}:}
This is an immediate application of part \eqref{thm:L1 of Lambda bounded}.  

\end{proof}

Let us consider the relationship between the norm $\HD^s$ and the classical $H^s$ norm.  

It is known (see \cite{CoIg.sqg} and \cite{CaSt}) that for $s \in (0,1)$ the spaces $\HD^s$ are equivalent to certain subsets of $H^s(\Omega)$ spaces defined in terms of the Gagliardo semi-norm.  In particular, we know that smooth functions with compact support are dense in $\HD^s$ and that elements of $\HD^s$ have trace zero for $s \in [\frac{1}{2},1]$.  

The most important fact for us is that the fractional Sobolev norms defined in terms of extension (for which we have access to a variety of theorems regarding compactness and Sobolev embeddings) are dominated by our $\HD^s$ norm with a constant that is independent of $\Omega$.  

Consider the extension-by-zero operator $E:L^2(\Omega) \to L^2(\R^2)$
\[ E f (x) = \begin{cases} f(x) & x \in \Omega, \\ 0 & x \in \R^2 \setminus \Omega. \end{cases} \]

\begin{proposition} \label{thm:hadamard 3 lines}
Let $\Omega \subseteq \R^2$ be any bounded open set with Lipschitz boundary.  For any $s \in [\frac{1}{2},1]$ and function $f \in \HD^s$,
\[ \int_{\R^2} \abs{\paren{-\Laplace}^{s/2} E f}^2 \leq \int_\Omega \abs{\Lambda^s f}^2. \]
Here $\paren{-\Laplace}^s$ is defined in the fourier sense.  
\end{proposition}

We will prove this proposition by interpolating between $s = 0$ and $s=1$.  Before we can do this, we must prove the same in the $s=1$ case.  

\begin{lemma} \label{thm:H1 and H1}
Let $\Omega \subseteq \R^2$ be any bounded open set with Lipschitz boundary.  For all functions $f$ in $\HD^1$,
\[ \int_\Omega \abs{\grad f}^2 = \int_\Omega \abs{\Lambda f}^2. \]
\end{lemma}

\begin{proof}
Let $\eigen{i}$ and $\eigen{j}$ be two eigenfunctions of the Dirichlet Laplacian on $\Omega$.  Note that these functions are smooth in the interior of $\Omega$ and vanish at the boundary, so we can apply the divergence theorem and find
\[ \int \grad \eigen{i} \cdot \grad \eigen{j} = - \int \eigen{i} \Laplace \eigen{j} = \lambda_j \int \eigen{i} \eigen{j} = \lambda_j \delta_{i=j}. \]
%Because $\Omega$ has Lipschitz boundary, and because $\eigen{i} \grad \eigen{j}$ is smooth on $\Omega$ and countinuous and bounded on $\overline{\Omega}$ vanishing on the boundary, therefore 
%\[ \int_\Omega \div(\eigen{i} \grad \eigen{j}) = \int_{\del \Omega} \eigen{i} \grad \eigen{j}. \]
%But $\eigen{i} \grad \eigen{j}$ vanishes on the boundary, so the right hand side vanishes.  Moreover, $\div(\eigen{i} \grad \eigen{j}) = \grad \eigen{i} \cdot \grad \eigen{j} + \eigen{i} \Laplace \eigen{j}$.  Therefore
%\[ \int \grad \eigen{i} \cdot \grad \eigen{j} = - \int \eigen{i} \Laplace \eigen{j} = \lambda_j \int \eigen{i} \eigen{j} = \lambda_j \delta_{i=j}. \]
%Of course, the inner product of two eigenfunctions is 0 unless they are the same eigenfunction, in which case it is 1.  

Consider a function $f = \sum f_k \eigen{k}$ which is an element of $\HD^1$, by which we mean $\sum \lambda_k f_k^2 < \infty$.  Since $\norm{\grad \eigen{k}}_{L^2(\Omega)} = \sqrt{\lambda_k}$, the following sums all converge in $L^2(\Omega)$ and hence the calculation is justified:
\begin{align*}
\int \abs{\grad f}^2 &= \int \paren{\sum_i f_i \grad \eigen{i} } \paren{\sum_j f_j \grad \eigen{j}}
\\ &= \int \sum_{i,j} (f_i f_j) \grad \eigen{i} \cdot \grad \eigen{j}
\\ &= \sum_{i,j} (f_i f_j) \int \grad \eigen{i} \cdot \grad \eigen{j}
\\ &= \sum_j \lambda_j f_j^2.
\end{align*}
From this the result follows.
\end{proof}


We come now to the proof of Proposition \ref{thm:hadamard 3 lines}.  
\begin{proof}
Let $g$ be any Schwartz function in $L^2(\R^2)$, and let $f$ be a function in $\HD^s$.  
%Let $E:\HD^1(\Omega) \to H^1(\R^2)$ be a the extension-by-zero operator, where $H^1$ denotes the classical Sobolev space defined using the gradient.  
Define the function
\[ \Phi(z) = \int_{\R^2} \paren{-\Laplace}^{z/2} g E \Lambda^{s-z} f, \qquad z \in \C, \Re(z) \in [0,1]. \]

Recall (see e.g. [citation Jerison, Kenig]) that when $t \in \R$, $(-\Laplace)^{i t}$ is a unitary transformation on $L^2(\R^2)$, and $\Lambda^{i t}$ is a unitary transformation on $L^2(\Omega)$.  

When $\Re(z) = 0$, then $\norm{\paren{-\Laplace}^{z/2} g}_2 = \norm{g}_2$ and $\norm{\Lambda^{s-z} f}_2 = \norm{f}_{\HD^s}$.  Hence
\[ \Phi(z) \leq \norm{g}_2 \norm{f}_{\HD^s}. \]

When $\Re(z)=1$, integrate by parts to obtain
\[ \Phi(z) = \int_{\R^2} \paren{-\Laplace}^{(z-1)/2} g  \paren{-\Laplace}^{1/2} E \Lambda^{s-z} f. \]
Then $\norm{\paren{-\Laplace}^{(z-1)/2} g}_2 = \norm{g}_2$, while $\norm{\Lambda^{s-z} f}_{\HD^1} = \norm{f}_{\HD^s}$.  As an $\HD^1$ function, $\Lambda^{s-z} f$ has trace zero so 
\[ \norm{\grad E \Lambda^{s-z} f}_{L^2(\R^2)} = \norm{\grad \Lambda^{s-z} f}_{L^2(\Omega)} = \norm{f}_{\HD^s}. \]
Of course $\norm{\paren{-\Laplace}^{1/2} \,\cdot\,}_{L^2(\R^2)} = \norm{\grad \,\cdot\,}_{L^2(\R^2)}$ in general so
\[ \Phi(z) \leq \norm{g}_2 \norm{f}_{\HD^s}. \]

In order to apply the Hadamard three-lines lemma, we must show that $\Phi$ is differentiable in the interior of its domain.  

Rewrite the integrand of $\Phi$ as
\[ \Four\n\paren{ |\xi|^z \hat{g} } E \sum_k \lambda_k^{\frac{s-z}{2}} f_k. \]
The derivative $\ddz$ commutes with linear operators like $\Four\n$ and $E$, so the derivative is
\begin{equation} \label{derivative of integrand} \Four\n\paren{ \ln(|\xi|) |\xi|^z \hat{g} } E \sum_k \lambda_k^{\frac{s-z}{2}} f_k + \Four\n\paren{ |\xi|^z \hat{g} } E \sum_k \frac{-1}{2} \ln(\lambda_k) \lambda_k^{\frac{s-z}{2}} f_k. \end{equation}

Fix some $z \in \C$ with $\Re(z) \in (0,1)$.  Since $g$ is a Schwartz function, $\ln(|\xi|)|\xi|^z \hat{g}$ is in $L^2$.  Moreover, for any $\eps>0$ we have $\ln(\lambda_k) \lambda_k^{\frac{s-z}{2}} \leq C \lambda_k^{\frac{s-z+\eps}{2}}$ for some $C$ independent of $k$ but dependent on $z$, $\eps$.  Take $\eps < \Re(z)$ and, since $f \in \HD^{s}$, this sum will converge in $L^2$.  

The differentiated integrand \eqref{derivative of integrand} is therefore a sum of two products of $L^2$ functions.  In particular it is integrable, which means we can interchange the integral sign and the derivative $\ddz$ and prove that $\Phi'(z)$ is finite for all $0<\Re(z)<1$. 

By the Hadamard three-lines lemma, for any $z \in (0,1)$ we have $\Phi(z) \leq \norm{g}_2 \norm{f}_{\HD^s}$.  
Evaluating $\Phi(s)$, we see
\[ \int_{\R^2} \paren{-\Laplace}^{s/2} g E f \leq \norm{g}_{L^2(\R^2)} \norm{f}_{\HD^s}. \]
This inequality holds for any Schwartz function $g \in L^2(\R^n)$ and any $f \in \HD^{s}$,

Since Schwartz functions are dense in $L^2(\R^2)$, the proof is complete.  
\end{proof}



%-------%-------%-------%-------%-------%-------%-------%-------%-------%-------%-------%-------



\section{$L^\infty$ bounds for $\theta$} \label{sec:Linfty}

First let us derive an energy inequality.  

\begin{lemma}[Caccioppoli Estimate] \label{thm:caccioppoli}
Let $\theta \in L^\infty(0,T; L^4(\Omega) \cap H^1(\Omega))$ and $u \in L^\infty(0,T; L^4(\Omega))$ solve \eqref{eq:main linear} in the sense of distributions.  Let $\Psi: [-T,0]\times \Omega \to \R$ be non-negative, Lipschitz-in-space, and H\"{o}lder continuous-in-space with exponent $\gamma < 1/2$.  Then the decomposition
\[ \theta = \theta_+ + \Psi - \theta_- \]
satisfies the inequality
\[ \ddt \int \theta_+^2 + \int \abs{\Lambda^{1/2} \theta_+}^2 - \chevron{\theta_+,\theta_-}_{1/2} \leq C \paren{ \int \indic{\theta_+ > 0} + \int \theta_+ (\del_t \Psi + u\cdot\grad\Psi) } \]
with the constant $C$ depending on $\norm{\grad \Psi}_\infty$ and $\sup_t \bracket{\Psi(t,\cdot)}_\gamma$.  

\end{lemma}

\begin{proof}
We multiply \eqref{eq:main linear} by $\theta_+$ and integrate in space to obtain
\[ 0 = \int \theta_+ \bracket{ \del_t + u \cdot \grad + \Lambda } \paren{\theta_+ + \Psi - \theta_-} \]
which decomposes into three terms, corresponding to $\theta_+$, $\Psi$, and $\theta_-$.  We analyze them one at a time.  

Firstly,
\begin{align*} 
\int \theta_+ \bracket{ \del_t + u \cdot \grad + \Lambda } \theta_+ &= (1/2) \ddt \int \theta_+^2 + (1/2) \int \div u \, \theta_+^2 + \int \abs{\Lambda^{1/2} \theta_+}^2
\\ &= (1/2) \ddt \int \theta_+^2 + \int \abs{\Lambda^{1/2} \theta_+}^2.
\end{align*}

The $\Psi$ term produces important error terms:
\begin{align*} 
\int \theta_+ \bracket{ \del_t + u \cdot \grad + \Lambda } \Psi &= \int \theta_+ \del_t \Psi + \int \theta_+ u \cdot \grad \Psi + \int \Lambda^{1/2} \theta_+ \Lambda^{1/2} \Psi
\\ &= \int \theta_+ (\del_t \Psi + u \cdot \grad \Psi) + \int \Lambda^{1/2} \theta_+ \Lambda^{1/2} \Psi
\end{align*}

Since $\theta_+$ and $\theta_-$ have disjoint support, the $\theta_-$ term is nonnegative by Lemma \ref{thm:disjoint}:
\begin{align*} 
\int \theta_+ \bracket{ \del_t + u \cdot \grad + \Lambda } \theta_- &= (1/2) \int \theta_+ \del_t \theta_- + \int \theta_+ u \cdot \grad \theta_- + \int \Lambda^{1/2} \theta_+ \Lambda^{1/2} \theta_-
\\ &= \int \Lambda^{1/2} \theta_+ \Lambda^{1/2} \theta_- \leq 0.
\end{align*}

Put together, we arrive at 
\[ (1/2) \ddt \int \theta_+^2 + \int \abs{\Lambda^{1/2} \theta_+}^2 - \int \Lambda^{1/2}\theta_+ \Lambda^{1/2} \theta_- + \int \Lambda^{1/2} \theta_+ \Lambda^{1/2} \Psi \leq \abs{\int \theta_+ (\del_t \Psi + u \cdot \grad \Psi) \cdot \grad \Psi}. \]

At this point we break down the $\Lambda^{1/2} \theta_+ \Lambda^{1/2} \Psi$ term using the formula from Proposition~\ref{thm:Caff Stinga representation}.  
\[ \int \Lambda^{1/2} \theta_+ \Lambda^{1/2} \Psi = \iint [\theta_+(x)-\theta_+(y)][\Psi(x)-\Psi(y)] K(x,y) + \int \theta_+ \Psi B. \]
Since $B \geq 0$ and $\Psi$ is non-negative by assumption, the $B$ term is non-negative and so
\[ \int \Lambda^{1/2} \theta_+ \Lambda^{1/2} \Psi \geq \iint [\theta_+(x)-\theta_+(y)][\Psi(x)-\Psi(y)] K(x,y). \]
The remaining integral is symmetric in $x$ and $y$, and the integrand is only nonzero if at least one of $\theta_+(x)$ and $\theta_+(y)$ is nonzero.  Hence
\[ \iint [\theta_+(x)-\theta_+(y)][\Psi(x)-\Psi(y)] K(x,y) \leq 2 \iint \indic{\theta_+>0}(x) \abs{\theta_+(x)-\theta_+(y)} \cdot \abs{\Psi_t(x)-\Psi_t(y)} K(x,y). \]
Now we can break up this integral using the Peter-Paul variant of H\"{o}lder's inequality.  
\[ \abs{\iint [\theta_+(x)-\theta_+(y)][\Psi(x)-\Psi(y)] K(x,y)} \leq \eps \int \abs{\Lambda^{1/2}\theta_+}^2 + \frac{1}{\eps} \iint \indic{\theta_+>0}(x) [\Psi(x)-\Psi(y)]^2 K(x,y). \]

It remains to bound the quantity $[\Psi(x)-\Psi(y)]^2 K(x,y)$.  By Proposition~\ref{thm:Caff Stinga representation}, there is a universal constant $C$ such that
\[ K(x,y) \leq \frac{C}{|x-y|^{3}}. \]
The cutoff $\Psi$ is Lipschitz, and H\"{o}lder continuous with exponent $\gamma < 1/2$ by assumption.  Therefore 
\[ [\Psi(x)-\Psi(y)]^2 K(x,y) \leq |x-y|^{-1} \wedge |x-y|^{2\gamma-3}. \]
Since $3-2\gamma > 2$, this quantity is integrable.  Thus
\[ \int \indic{\theta_+>0}(x) \int [\Psi(x)-\Psi(y)]^2 K(x,y) \,dxdy \leq C(\norm{\Psi}_\textrm{Lip}, \bracket{\Psi}_\gamma) \int \indic{\theta_+>0} \,dx. \]
Combining [citation, like 4 different things are combined] we arrive at
\[ \ddt \int \theta_+^2 + \int \abs{\Lambda^{1/2} \theta_+}^2 - \chevron{\theta_+,\theta_-}_{1/2} \lesssim \int \theta_+ (\del_t\Psi+u\cdot\grad\Psi) + \int \indic{\theta_+>0}.\]
\end{proof}

This is sufficient to prove that a solution to \eqref{eq:main linear} with $L^2$ initial data has bounded $L^\infty$-norm after any small time.  

\begin{proposition}[$L^2$ to $L^\infty$] \label{thm:L2 to Linfty}
If $\theta$ and $u$ solve \eqref{eq:main linear} on $[0,T] \times \Omega$ and $\theta_0 \in L^2$, then for any time $S \in (0,T)$ there exists a constant $C = C(S)$ such that
\[ \norm{\theta}_{L^\infty([S,T]\times \Omega)} \leq C \norm{\theta_0}_{L^2(\Omega)}. \]
\end{proposition}

\begin{proof}
It is trivial to show that the $L^2(\Omega)$ norm of any smooth solution $\theta$ of \eqref{eq:main linear} does not increase in time.  Simply multiply the function by $\theta$ and integrate.  

Moreover, using Lemma \ref{thm:caccioppoli} with $\Psi(t,x) = \norm{\theta(T,\cdot)}_{L^\infty(\Omega)}$ tells us that the $L^\infty(\Omega)$ norm of a solution, once finite, is non-increasing in time.

To show that the $L^\infty(\Omega)$ norm of a solution with $L^2(\Omega)$ initial data is bounded after a small time, consider the sequence of functions
\[ \theta_k := (\theta(t,x) - 1 + 2^{-k})_+ \]
and define
\[ \E_k := \int_{-1-2^{-k}}^0 \int_\Omega \theta_k^2 \,dxdt. \]

When $\theta_{k+1}>0$, then in particular $\theta_k \geq 2^{-k-1}$.  Thus for any finite $p$, there exists a constant $C$ so
\[ \indic{\theta_{k+1}>0} \leq C^k \theta_k^p. \]
In particular,
\[ \E_{k+1} \leq C^k \int_{-1-2^{-k}}^0 \int \theta_k^3. \]

Applying the energy inequality $\theta$, $\phi$, and $\Gamma$ we obtain
\[ \sup_{-1-2^{-k-1}<t<0} \int \theta_{k+1}^2 + \int_{-1-2^{-k-1}}^0 \int \abs{\Lambda^{1/2}\theta_{k+1}}^2 \leq C^k \int_{-1-2^{-k}}^0 \theta_k^2 = \E_k. \]

However, by Sobolev embedding and the fact that $\HD^{1/2}$ controls classical $H^{1/2}$ controls $L^4$,
\[ \norm{\theta_{k+1}}_{L^3([-1-2^{-k-1},0]\times\Omega)} \leq C^k \E_k^{1/2}. \]

Therefore
\[ \E_{k+1} \leq C^k \E_k^{3/2}. \]

It follows by a well known result [citation] that for $\E_0$ sufficiently small (say less than $\bar{C}$), $\E_k \to 0$ as $k \to \infty$.  

Notice that, since the $L^2(\Omega)$ norm of $\theta$ does not increase in time,
\[ \E_0 = \int_{-2}^0 \int_\Omega (\theta)_+ \,dxdt \leq 2 \int \theta_0^2 \,dx . \]
Moreover, as $k \to \infty$ we have
\[ \E_k \to \int_{-1}^0 \int_\Omega (\theta - 1)_+ \,dxdt \]

Thus, if $\norm{\theta_0}_{L^2(\Omega)} \leq \sqrt{\bar{C}/2}$ then $\theta \leq 1$ on $[-1,0]$.  

Since \eqref{eq:main linear} is linear and scales in time and space as in Lemma \ref{thm:scaling} (and since the constant $\bar{C}$ does not depend on $\Omega$), we can take a solution $\theta$ with arbitrary initial $L^2$ norm and apply this result to a scaled version. 

The result follows.  
\end{proof}


%-------%-------%-------%-------%-------%-------%-------%-------%-------%-------%-------%-------

\section{Littlewood-Paley Theory} \label{sec:littlewood paley}

In this section we will prove that, because $\theta$ is uniformly bounded in $L^\infty$, the velocity $u = \grad^\perp \Lambda^{-1} \theta$ is calibrated (see Definition~\ref{def:calibrated}).  The proof will utilize a Littlewood-Paley theory adapted to a bounded set $\Omega$.  

Let $\phi$ be a Schwartz function on $\R$ which is suited to Littlewood-Paley decomposition.  Specifically, $\phi$ is non-negative, supported on $[1/2,2]$, and has the property that
\[ \sum_{j \in \Z} \phi(2^j \xi) = 1 \qquad \forall \xi \neq 0. \]
This allows us to define the Littlewood-Paley projections.  For any $f = \sum f_k \eigen{k}$ in $L^2(\Omega)$
\[ P_j f := \sum_k \phi(2^j \lambda_k^{1/2}) f_k \eigen{k}. \]
Note that $P_j$ depends strongly on the domain $\Omega$.  

Recall that $-\Laplace_D$ has some smallest eigenvalue $\lambda_0$ (depending on $\Omega$) so if we define $j_0 = \log_2(\lambda_0)-1$ then $P_j = 0$ for all $j < j_0$.

Our goal in this section is to prove the following proposition:

\begin{proposition} \label{thm:u is calibrated}
Let $\Omega \subseteq \R^2$ be a bounded set with $C^{2,\beta}$ boundary for some $\beta \in (0,1)$.  Let $\theta \in L^\infty(\Omega)$.  Then there exists a sequence of divergence-free functions $(u_j)_{j \in \Z}$ calibrated for some constant $\kappa = \kappa(\Omega, \norm{\theta}_\infty)$ with center 0 (see Definition~\ref{def:calibrated}) such that
\[ \grad^\perp \Lambda^{-1} \theta = \sum_{j \in \Z} u_j \]
with the infinite sum converging in the sense of $L^2$.  

Moreover there exists some $j_0 \in \Z$ such that $u_j = 0$ for all $j < j_0$.  
\end{proposition}

Before we can prove this, we state a few important lemmas.  

The Bernstein Inequalities adapted for a bounded domain are proved in \cite{IMTb}.  We restate their result here:
\begin{lemma}[Bernstein Inequalities] \label{thm:IMT stuff}
Let $1 \leq p \leq \infty$, and $\Omega \subset \R^2$ a bounded open set with $C^{2,\beta}$ boundary for some $\beta \in (0,1)$, and let $(P_j)_{j \in \Z}$ be a Littlewood-Paley decomposition as defined above.  

There exists a constant $C$ depending on $p$ and $\Omega$ such that the following hold for any $f \in L^p(\Omega)$:

For any $\alpha \in \R$ and $j \in \Z$, 
\[ \norm{\Lambda^\alpha P_j f}_{L^p(\Omega)} \leq C 2^{\alpha j} \norm{f}_{L^p(\Omega)}. \]

For any $\alpha \in \R$ and $j \geq j_0$
\[ \norm{\grad \Lambda^\alpha P_j f}_{L^p(\Omega)} \leq C 2^{(1+\alpha) j} \norm{f}_{L^p(\Omega)}. \]
\end{lemma}

\begin{proof}
The first claim is Lemma 3.5 in \cite{IMTb}.  It is also an immediate corollary of \cite{IMTs} Theorem 1.1.  

The second claim is similar to Lemma 3.6 in \cite{IMTb}.  A hypothesis of Lemma 3.6 is that
\[ \norm{\grad e^{-t\Laplace_D}}_{L^\infty \to L^\infty} \leq \frac{C}{\sqrt{t}} \qquad 0 < t \leq 1 \]
(a property of $\Omega$).  The result of Lemma 3.6 only covers the case $j > 0$.  

In \cite{FMP} it is proved that that if $\Omega$ is $C^{2,\alpha}$ then
\[ \norm{\grad e^{-t\Laplace_D}}_{L^\infty \to L^\infty} \leq \frac{C}{\sqrt{t}} \qquad 0 < t \leq T \]
which, by taking some $T$ depending on $j_0$, is enough to prove the desired result for $j \geq j_0$ by a trivial modification of the proof in \cite{IMTb}.  
\end{proof}

The following lemma is a simple but crucial result which can be thought of as describing the commutator of the gradient operator and the projection operators.  In the case of $\R^2$, the Littlewood-Paley projections commute with the gradient so $P_i \grad P_j = 0$ unless $|i-j|\leq 1$.  On a bounded domain, this is not the case.  One can say that the gradient does not maintain localization in frequency-space.  However, the following lemma formalizes the observation that $P_i \grad P_j \approx 0$ when $i << j$.  

\begin{lemma} \label{thm:grad and proj}
There exists a constant $C$ depending on $\Omega$ such that or any function $f \in L^\infty(\Omega)$,
\[ \norm{P_i \grad P_j f}_\infty \leq C \min(2^j,2^i) \norm{f}_\infty. \]
\end{lemma}
\begin{proof}
Let $g$ be an $L^1$ function.  Then since $P_i$ is self-adjoint
\[ \int g P_i \grad P_j f = \int (P_i g) \grad P_j f \leq C 2^j \norm{g}_1 \norm{f}_\infty \]
by Lemma \ref{thm:IMT stuff}.  

Further integrating by parts,
\[ \int g P_i \grad P_j f = - \int (\grad P_i g) P_j f \leq C 2^i \norm{g}_1 \norm{f}_\infty. \]
This also follows from Lemma \ref{thm:IMT stuff}.  

The result follows.  
\end{proof}


This final lemma allows us to interpolate using H\"{o}lder seminorms.  The results are not presumed to be novel, but since their proofs were difficult to find in the literature we include them below.  

\begin{lemma} \label{thm:Holder interpolation}
Let $\alpha \in (0,1)$.  There exists a constant $C = C(\alpha)$ such that, for any set $\Omega$ and any $f \in C^{0,1}(\Omega)$,
\[ \bracket{f}_\alpha \leq C \norm{f}_\infty^{1-\alpha} \norm{\grad f}_\infty^\alpha. \]

\vspace{5mm}

Let $\alpha \in (0,1)$ and $\Omega$ a set that satisfies the cone condition.  There exist constants $C = C(\alpha, \Omega)$ and $\ell = \ell(\Omega)$ such that, for any $f \in C^{2,\alpha}(\Omega)$
\[ \norm{D^2 f}_\infty \leq C \paren{ \delta\n \norm{\grad f}_\infty  + \delta^\alpha \bracket{D^2 f}_\alpha }\]
for all $\delta < \ell$.  
\end{lemma}

We are now ready to prove Proposition \ref{thm:u is calibrated}.  

\begin{proof}
For each $j \in \Z$, we define $u_j$ to be the $\frac{\pi}{2}$-rotation of the Riesz transform of the $j\ith$ Littlewood-Paley projection of $\theta$:
\[ u_j := \grad^\perp \Lambda^{-1} P_j \theta. \]
Qualitatively, we know that $\theta \in L^2$ and hence $u_j \in L^2$.  In fact, $u = \sum u_j$ in the $L^2$ sense.  

We must bound $u_j$, $\Lambda^{-1/4} u_j$, and $\grad u_j$ all in $L^\infty(\Omega)$.  

By straightforward application of Lemma~\ref{thm:IMT stuff},
\begin{equation} \label{uj in Linfty} \norm{u_j}_\infty \leq C \norm{\theta}_\infty. \end{equation}

Since $u_j \in L^2$, we know that
\[ \Lambda^{-1/4} u_j = \sum_{i \in \Z} P_i \Lambda^{-1/4} u_j. \]
Define $\bar{P}_k := P_{k-1} + P_k + P_{k+1}$.  Then $\bar{P}_k P_k = P_k$, and since the projections $P_k$ are spectral operators, they commute with $\Lambda^s$.  We therefore rewrite
\[ \paren{P_i \Lambda^{-1/4} u_j}^\perp = \paren{\Lambda^{-1/4} \bar{P}_i} P_i \grad P_j \paren{\Lambda^{-1} \bar{P}_j} \theta. \]
On the right hand side we have three bounded linear operators applied sequentially to $L^\infty$.  The first operator has norm $C 2^{-j}(2^{1}+2^0+2^{-1})$ by Lemma~\ref{thm:IMT stuff}.  The second operator has norm $C \min(2^j, 2^i)$ by Lemma~\ref{thm:grad and proj}.  The third operator has norm $C 2^{-i/4}(2^{1/4} + 2^0 + 2^{-1/4})$ by Lemma~\ref{thm:IMT stuff}.  (Of course, the perp operator is an isometry.)  Therefore
\[ \norm{ P_i \Lambda^{-1/4} u_j}_\infty \leq C 2^{-i/4} \min(2^j, 2^i) 2^{-j} \norm{\theta}_\infty. \]

Summing these bounds on the projections of $\Lambda^{-1/4} u_j$, and noting that
\[ \sum_{i \in \Z} 2^{-j} 2^{-i/4} \min(2^j,2^i) = 2^{-j} \sum_{i \leq j} 2^{i 3/4} + \sum_{i>j} 2^{-i/4} \leq C 2^{-j/4}, \]
we obtain
\begin{equation}\label{uj in W-1/4} \norm{\Lambda^{-1/4} u_j}_\infty \leq C 2^{-j/4} \norm{\theta}_\infty. \end{equation}

Lastly, we must show that $\grad u_j$ is in $L^\infty$.  Equivalently, we will show that $\Lambda^{-1} P_j \theta$ is $C^{1,1}$.  The method of proof is Schauder theory.  

For convenience, define
\[ F := \Lambda^{-1} P_j \theta. \]
Notice that $F$ is a linear combination of Dirichlet eigenfunctions, so in particular it is smooth and vanishes at the boundary.  Therefore
\[ -\Laplace F = \Lambda^2 F = \Lambda P_j \theta. \]


%For convenience, define 
%\[ F := \Lambda^{-1} P_j \theta \]
%and recall that $F$ is a finite linear combination of Dirichlet eigenfunctions, so in particular it is smooth and vanishes at the boundary.  Moreover, its Laplacian is 
%\[ f := \Laplace F = \Lambda P_j \theta \]
%which is also smooth and vanishes at the boundary.



We apply the standard Schauder estimate from \cite{GiTr} Theorem 6.6 to bound some $C^{2,\alpha}$ semi-norm of $F$ by the $L^\infty$ norm of $F$ and the $C^\alpha$ norm of its Laplacian.  By assumption there exists $\beta \in (0,1)$ such that $\Omega$ is $C^{2,\beta}$, and for this $\beta$ we have by the Schauder estimate
\begin{equation} \label{schauder estimate} \bracket{D^2 F}_{\beta} \leq C \norm{\Lambda^{-1} P_j \theta}_\infty + C \norm{\Lambda P_j \theta}_\infty + C \bracket{\Lambda P_j \theta}_{\beta}. \end{equation}

By Lemma \ref{thm:IMT stuff},
\begin{align*} 
\norm{\Lambda^{-1} P_j \theta}_\infty &\leq C 2^{-j} \norm{\theta}_\infty, \\
\norm{\Lambda P_j \theta}_\infty &\leq C 2^j \norm{\theta}_\infty, \\
\norm{\grad \Lambda P_j \theta}_\infty &\leq C 2^{2j} \norm{\theta}_\infty. 
\end{align*}
%\begin{align*} 
%\norm{F}_\infty &= \norm{\Lambda^{-1} P_j \theta}_\infty \leq C 2^{-j} \norm{\theta}_\infty, \\
%\norm{f}_\infty &= \norm{\Lambda P_j \theta}_\infty \leq C 2^j \norm{\theta}_\infty, \\
%\norm{\grad f}_\infty &= \norm{\grad \Lambda P_j \theta}_\infty \leq C 2^{2j} \norm{\theta}_\infty. 
%\end{align*}
By Lemma~\ref{thm:Holder interpolation} we can interpolate these last two bounds to obtain
\[ \bracket{\Lambda P_j \theta}_\beta \leq C 2^{j(1+\beta)} \norm{\theta}_\infty. \]
Plugging these estimates into \eqref{schauder estimate} yields
\[ \bracket{D^2 F}_\beta \leq C \paren{2^{-j} + 2^j + 2^{j(1+\beta)}} \norm{\theta}_\infty. \]

Recall that without loss of generality we can assume $j \geq j_0$.  Therefore up to a constant depending on $j_0$, the term $2^{j(1+\beta)}$ bounds $2^j$ and $2^{-j}$ so we can write
\[ \bracket{D^2 F}_\alpha \leq C 2^{j(1+\beta)} \norm{\theta}_\infty. \]

Using this estimate and the fact that $\norm{\grad F}_\infty = \norm{\grad \Lambda^{-1} P_j \theta}_\infty \leq C \norm{\theta}_\infty$ (see \eqref{uj in Linfty}), we can use Lemma~\ref{thm:Holder interpolation} to interpolate.  For some constant $\ell$ depending on $\Omega$, for any $\delta \leq \ell$ we have
\begin{align*}
\norm{D^2 F}_\infty &\leq C \paren{\delta\n \norm{\grad F}_\infty + \delta^{\beta} \bracket{D^2 F}_\beta}
\\ &\leq C \paren{\delta\n + \delta^\beta 2^{j(1+\beta)}} \norm{\theta}_\infty.  
\end{align*}

Set $\delta = 2^{-j} (2^{j_0} \ell) \leq \ell$.  Then
\[ \bracket{D^2 F}_\infty \leq C \paren{C 2^j + 2^{-j\beta} 2^{j(1+\beta)}} \norm{\theta}_\infty = C(\Omega) 2^j \norm{\theta}_\infty. \]

Since $D^2 F = \grad u_j$, this estimate together with \eqref{uj in Linfty} and \eqref{uj in W-1/4} complete the proof.  
\end{proof}

Now that we know that $u$ is calibrated, we end the section with a final lemma to show why calibrated sequences are useful.  






%-------%-------%-------%-------%-------%-------%-------%-------%-------%-------%-------%-------

\section{De Giorgi Estimates} \label{sec:de giorgi}

Our goal in this section is to prove De Giorgi's first and second lemmas for solutions to \eqref{eq:main linear} with $u$ uniformly calibrated.  

The first step is to show that a calibrated velocity can be rewritten in a more practical format:
\begin{lemma} \label{thm:calibration is good}
Let 
\[ u = \sum_{j_0}^\infty u_j \]
with the sum converging in the $L^2$ sense.  Assume that $(u_j)_{j \in \Z}$ is a calibrated sequence with constant $\kappa$ and some center, and that $\div(u_j)=0$.  

Then
\[ u = \ulow + \uhigh \]
with 
\begin{align*} 
\norm{\grad \ulow}_{L^\infty([-T,0]\times \Omega)} &\leq 2 \kappa, \\
\norm{\Lambda^{-1/4} \uhigh}_{L^\infty([-T,0]\times\Omega)} &\leq 6 \kappa. 
\end{align*}
and $\div(\ulow) = \div(\uhigh) = 0$.  
\end{lemma}

We call $\ulow$ the low-pass term, and $\uhigh$ the high-pass term.  

\begin{proof}
Let $N$ be the center to which $(u_j)_{j \in \Z}$ is calibrated.  

We define
\[ \uhigh = \sum_{j > N} u_j \]
and bound
\[ \norm{\Lambda^{-1/4} \uhigh}_\infty \leq \sum_{j>N} \norm{\Lambda^{-1/4} u_j}_\infty \leq \kappa \frac{2^{-1/4}}{1-2^{-1/4}}. \]

We define
\[ \ulow = \sum_{j = j_0}^N u_j. \]
and bound
\[ \norm{\grad \ulow}_\infty \leq \sum_{j \leq N} \norm{\grad u_j}_\infty \leq \kappa \frac{1}{1 - 2^{-1}}. \]
\end{proof}

In order to prove the De Giorgi lemmas, we must derive an energy inequality for the function $\paren{\theta - \Psi}_+$ where $\Psi(t,x)$ is not constant, but rather grows sublinearly in $|x|$.  However, applying Lemma~\ref{thm:caccioppoli} to such a function, we see that its derivatives are only small if the quantity $\del_t \Psi + u \cdot \grad\Psi$ is bounded.  

To that end, we shall consider a family of functions $\theta$, $\ulow$, and $\uhigh:[-T,0]\times \Omega \to \R$, and paths $\Gamma$ and $\gamma:[-5,0] \to \R^2$ which satisfy
\begin{equation} \label{eq:main linear brokendown} \begin{cases}
\del_t \theta + (\ulow + \uhigh) \cdot \grad \theta + \Lambda \theta = 0 & \textrm{on } [-T,0] \times \Omega, \\
\div( \ulow) = \div(\uhigh) = 0 & \textrm{on } [-T,0]\times \Omega, \\
\dot{\Gamma}(t) + \dot{\gamma}(t) = \ulow(t, \gamma(t) + \Gamma(t)) & \textrm{on } [-T,0], \\
\gamma(0) = \Gamma(0) = 0.
\end{cases} \end{equation}

Here it is implicitly assumed that $\gamma(t) + \Gamma(t) \in \Omega$.  Generally speaking $\ulow$ and $\gamma$ will be Lipschitz functions while $\uhigh$ is merely in a weak space $W^{-1/4,\infty}$ and $\Gamma$ will trace out points in $\Omega$ where $\theta$ is well behaved by assumption.  See Section~\ref{sec:holder} for the construction of these functions $\Gamma$ and $\gamma$.  

Now we prove an energy inequality for solutions to \eqref{eq:main linear brokendown}.  

\begin{lemma}[Energy inequality] \label{thm:energy inequality}
Let $\kappa$, $\Comega$, $\Cgamma$, $T$, and $R$ be positive constants, and let $\psi:\R^2 \to \R$ be a function with bounded first and second derivatives. Then there exists a constant $C>0$ such that the following holds:

Let $\Omega \subseteq \R^2$ be a bounded open set with $C^{2,\beta}$ boundary for some $\beta \in (0,1)$.  Assume that Lemma~\ref{thm:Caff Stinga representation} hold on $\Omega$ with kernels that satisfy
\[ K_{1/4}(x,y) \leq \Comega |x-y|^{1/2} K_{1}. \]

Let $\theta$, $\ulow$, $\uhigh$, $\Gamma$ and $\gamma$ solve \eqref{eq:main linear brokendown} on $[-T,0]\times\Omega$, and satisfy $\norm{\Lambda^{-1/4} \uhigh}_{L^\infty([-T,0]\times\Omega)} \leq 6 \kappa$, $\norm{\grad \ulow}_{L^\infty([-T,0]\times\Omega)} \leq 2\kappa$, and $\norm{\dot{\gamma}}_{L^\infty([-T,0])} \leq \Cgamma$.  

Let $\psi:\R^2 \to \R$ have bounded derivative and bounded second derivative, and consider the functions
\[ \theta_+ := \paren{\theta - \psi(\cdot-\Gamma)}_+, \qquad \theta_- := \paren{\psi(\cdot-\Gamma) - \theta}_+. \]

If $\theta_+$ is supported on $x \in \Omega \cap B_R(\Gamma(t))$ for some $R>0$, then $\theta_+$ and $\theta_-$ satisfy the inequality
\[ \ddt \int \theta_+^2 + \int \abs{\Lambda^{1/2} \theta_+}^2 - \chevron{\theta_+,\theta_-}_{1/2} \leq C \paren{ \int \indic{\theta_+ > 0} + \int \theta_+ + \int \theta_+^2 }. \]
%with the constant $C$ depending on $\Cgamma$ and $T$, on $\norm{\Lambda^{-1/4} \uhigh}_\infty$, $\bracket{\ulow}_{3/4}$, and $\norm{\ulow}_\Lip$, and on $\norm{D^2 \phi}_\infty$, $\norm{\grad\phi}_\infty$, and $\sup \norm{|x|^{3/4} \grad\phi(x)}_\infty$.  
\end{lemma}

\begin{proof}
Define 
\[ \Psi(t,x) := \psi(x - \Gamma(t)) \]
so that
\[ \del_t \Psi + (\ulow + \uhigh)\cdot\grad \Psi = (\ulow - \dot{\Gamma} + \uhigh)\cdot \grad \psi(x-\Gamma(t)). \]

Applying Lemma \ref{thm:caccioppoli} to $\theta$ and $\Psi$ we arrive at
\[ \ddt \int \theta_+^2 + \int \abs{\Lambda^{1/2} \theta_+}^2 - \chevron{\theta_+,\theta_-}_{1/2} \leq C \paren{ \int \indic{\theta_+ > 0} + \int \theta_+ (\ulow-\dot{\Gamma}(t) + \uhigh) \cdot \grad\psi(x-\Gamma(t)) }. \]

Consider first the high-pass term $\int \theta_+ \uhigh\cdot\grad\psi$.  By inserting $\Lambda^{1/4}\Lambda^{-1/4}$ and then integrating by parts, we can apply Lemma~\ref{thm:Lambda stuff} parts \eqref{thm:L1 of Lambda1/4 bounded} and \eqref{thm:extra product rule} to obtain
\[ \int \Lambda^{-1/4} \uhigh \Lambda^{1/4} (\theta_+ \grad\psi) \leq C \norm{\Lambda^{-1/4} \uhigh}_\infty \paren{\norm{\grad\psi}_\infty + \norm{D^2 \psi}_\infty} |\supp(\theta_+)|^{1/2} \paren{ \norm{\theta_+}_{L^2} + \norm{\theta_+}_{\HD^{1/2}}}. \]
From H\"{o}lder's inequality with Peter-Paul, we obtain
%\[ \int \Lambda^{-1/4} \uhigh \Lambda^{1/4} (\theta_+ \grad\psi) \leq C(\phi,\eps) \norm{\Lambda^{1/4} \uhigh}_\infty \paren{\int \theta_+ + |\{\theta_+>0\}| + \int \theta_+^2 + \eps\n |\{\theta_+ > 0\}| + \eps \int \abs{\Lambda^{1/2} \theta_+}^2 }. \]
\[ \int \uhigh \theta_+ \grad \psi(x - \gamma(t)) \,dx \leq C(\psi,\eps) \kappa \paren{\int \indic{\theta_+>0} + \int \theta_+^2} + \eps \int \abs{\Lambda^{1/2} \theta_+}^2. \]

Consider now the low-pass term.  Recall that
\[ \dot{\Gamma} + \dot{\gamma} = \ulow(t, \Gamma+\gamma) \]
so
\[ \ulow(t,x) - \dot{\Gamma}(t) = \ulow(t,x) - \ulow(t,\Gamma+\gamma) + \dot{\gamma}. \]

By assumption, $|\dot{\gamma}|\leq \Cgamma$ and so for $t \in [-T,0]$ we have $|\gamma(t)| \leq T \Cgamma$.  

Since $\ulow$ is has bounded derivative,
\begin{align*} 
|\ulow(t,x) - \ulow(t,\Gamma+\gamma)| &\leq  |\ulow(t,x)-\ulow(t,\Gamma)| + |\ulow(t,\Gamma) - \ulow(t,\Gamma+\gamma)| 
\\ &\leq 2\kappa |x-\Gamma| + 2\kappa T \Cgamma. 
\end{align*}

Plugging these bounds into [cite] we obtain
\[ \abs{\ulow(t,x) - \dot{\Gamma}(t)} \leq (1+2\kappa T) \Cgamma + 2\kappa |x-\Gamma|. \]

Now we can bound the low pass term
%\begin{align*}
\[ \int (\ulow - \dot{\Gamma}) \theta_+ \grad\psi(x-\Gamma) \leq (1+2\kappa T) \Cgamma \norm{\grad\psi}_\infty \int \theta_+ \,dx +  \norm{\grad\psi}_\infty 2\kappa \int |x-\Gamma| \theta_+ \,dx. \]
%\\ &\leq (1+\norm{\grad\ulow}_\infty T) \Cgamma \norm{\grad\psi}_\infty \int \theta_+ \,dx +  \bracket{\ulow}_{3/4} \norm{|x|^{3/4} \grad\psi(x)}_\infty \int \theta_+ \,dx.
%\end{align*}
By assumption, $|x-\Gamma| \theta_+ \leq R \theta_+$, so the result follows.  
\end{proof}

This energy inequality is sufficient to apply the method of De Giorgi.  

\begin{lemma}[First De Giorgi Lemma] \label{thm:DG1}

Let $\kappa$, $\Comega$, and $\Cgamma$, be positive constants. Then there exists a constant $\delta_0>0$ such that the following holds:

Let $\Omega \subseteq \R^2$ be a bounded open set with $C^{2,\beta}$ boundary for some $\beta \in (0,1)$.  Assume that Lemma~\ref{thm:Caff Stinga representation} hold on $\Omega$ with kernels that satisfy
\[ K_{1/4}(x,y) \leq \Comega |x-y|^{1/2} K_{1}. \]

Let $\theta$, $\ulow$, $\uhigh$, $\Gamma$ and $\gamma$ solve \eqref{eq:main linear brokendown} on $[-2,0]\times\Omega$, and satisfy $\norm{\Lambda^{-1/4} \uhigh}_{L^\infty([-2,0]\times\Omega)} \leq 6 \kappa$, $\norm{\grad \ulow}_{L^\infty([-2,0]\times\Omega)} \leq 2\kappa$, and $\norm{\dot{\gamma}}_{L^\infty([-2,0])} \leq \Cgamma$.  

If
\[ \theta(t,x) \leq 2 + \paren{|x-\Gamma(t)|^{1/4}-2^{1/4}}_+ \qquad \forall t\in[-2,0], x \in \Omega \setminus B_2(\Gamma(t)) \]
and
\[ \int_{-2}^0 \int_{\Omega\cap B_2(\Gamma(t))} (\theta)_+^2 \,dxdt \leq \delta_0 \]
then
\[ \theta(t,x) \leq 1 \qquad \forall t \in [-1,0], x \in \Omega \cap B_1(\Gamma(t)). \]

\end{lemma}

\begin{proof}
Let $\psi$ be such that $\psi = 0$ on $B_1$ and $\psi(x) \geq 2 + \paren{|x|^{1/4}-2^{1/4}}_+$ for $|x|>2$ while $\psi$ is Lipschitz and $C^2$ and its gradient decays like $|x|^{-3/4}$.  

Consider the sequence of functions
\[ \theta_k := (\theta(t,x) - \psi(x - \Gamma(t)) - 1 + 2^{-k})_+ \]
and define
\[ \E_k := \int_{-1-2^{-k}}^0 \int_\Omega \theta_k^2 \,dxdt. \]

Notice that
\[ \E_0 = \int_{-2}^0 \int_\Omega (\theta - \psi(x-\Gamma))_+^2 \,dxdt \leq \delta_0. \]
Moreover, as $k \to \infty$ we have
\[ \E_k \to \int_{-1}^0 \int_\Omega (\theta - \psi(x-\Gamma) - 1)_+^2 \,dxdt \]
so in particular, if we can show $\E_k \to 0$ then $\theta \leq 1$ for $t \in [-1,0]$ and $x \in B_1(\Gamma)$.  

That's enough setup, let's argue that $\E_k \to 0$.  Notice that when $\theta_{k+1}>0$, then in particular $\theta_k \geq 2^{-k}$ [or something similar].  Thus for any finite $p$, there exists a constant $C$ so
\[ \indic{\theta_{k+1}>0} \leq C^k \theta_k^p. \]
In particular,
\[ \E_{k+1} \leq C^k \int_{-1-2^{-k}}^0 \int \theta_k^3. \]

Applying the energy inequality $\theta$, $\psi$, and $\Gamma$ we obtain
\[ \sup_{-1-2^{-k-1}<t<0} \int \theta_{k+1}^2 + \int_{-1-2^{-k-1}}^0 \int \abs{\Lambda^{1/2}\theta_{k+1}}^2 \leq C^k \int_{-1-2^{-k}}^0 \int \theta_k^2 = \E_k. \]

However, by Sobolev embedding and the fact that $\HD^{1/2}$ controls classical $H^{1/2}$ controls $L^4$, we know from Reisz-Thorin that the left side of the energy inequality controls the $L^3$ norm of $\theta_{k+1}$ so
\[ \norm{\theta_{k+1}}_{L^3([-1-2^{-k-1},0]\times\Omega)} \leq C^k \E_k^{1/2}. \]

Therefore
\[ \E_{k+1} \leq C^k \E_k^{3/2}. \]

It follows by a well known result [citation] that for $\E_0$ sufficiently small (say less than $\delta_0$), $\E_k \to 0$ as $k \to \infty$ which we already established is sufficient to obtain our result.  
\end{proof}


This is coming along quite nicely.  We can move on to DG2, the isoperimetric inequality.  

\begin{lemma}[Second De Giorgi Lemma] \label{thm:DG2}
Let $\kappa$, $\Comega$, and $\Cgamma$, be positive constants. Then there exists a constant $\mu>0$ such that the following holds:

Let $\Omega \subseteq \R^2$ be a bounded open set with $C^{2,\beta}$ boundary for some $\beta \in (0,1)$.  Assume that Lemma~\ref{thm:Caff Stinga representation} hold on $\Omega$ with kernels that satisfy
\[ K_{1/4}(x,y) \leq \Comega |x-y|^{1/2} K_{1}. \]

Let $\theta$, $\ulow$, $\uhigh$, $\Gamma$ and $\gamma$ solve \eqref{eq:main linear brokendown} on $[-2,0]\times\Omega$, and satisfy $\norm{\Lambda^{-1/4} \uhigh}_{L^\infty([-2,0]\times\Omega)} \leq 6 \kappa$, $\norm{\grad \ulow}_{L^\infty([-2,0]\times\Omega)} \leq 2\kappa$, and $\norm{\dot{\gamma}}_{L^\infty([-2,0])} \leq \Cgamma$.  

Suppose that for $t \in [-5,0]$ and any $x \in \Omega$,
\[ \theta(t,x) \leq 2 + \paren{|x-\Gamma(t)|^{1/4}-2^{1/4}}_+. \]

Then the three conditions
\begin{align} 
\abs{\{\theta \geq 1\} \cap [-2,0]\times B_2(\Gamma)} &\geq \delta_0/4, \label{mass assumption above} \\
\abs{\{0 < \theta < 1\} \cap [-4,0]\times B_4(\Gamma)} &\leq \mu, \nonumber \\
\abs{\{\theta \leq 0\} \cap [-4,0]\times B_4(\Gamma)} &\geq 2 |B_4| \label{mass assumption below}
\end{align}
cannot simultaneously be met.  
\end{lemma}

Here $\delta_0$ is the constant from Lemma~\ref{thm:DG1}, which of course depends on $\kappa$, $\Cgamma$, and $\Comega$.  

\begin{proof}
Suppose that the theorem is false.  Then there must exist, for each $n \in \N$, a bounded $C^{2,\alpha}$ set $\Omega_n$ and function $\theta_n: [-5,0]\times \Omega_n \to \R$, functions $\ulow^n, \uhigh^n: [-5,0]\times\Omega_n \to \R^2$, and paths $\Gamma_n,\gamma_n:[-5,0]\to\R^2$ which solve \eqref{eq:main linear brokendown} and satisfy all of the the assumptions of our lemma (with the same constants $\kappa$, $\Cgamma$, and $\Comega$), except that
\begin{equation} \label{mass assumption between} \abs{\{0 < \theta_n < 1\} \cap [-4,0]\times B_4(\Gamma_n)} \leq 1/n. \end{equation}

Let $\psi:\R^2 \to \R$ be a smooth function which vanishes on $B_2$ such that $\psi(x) = 2 + \paren{|x|^{1/4}-2^{1/4}}_+$ for $|x|>3$.  

Fix $n$ and define 
\[ \theta_+ := \paren{\theta_n - \psi(x-\Gamma_n)}_+. \]
Then $\theta_+$ is supported on $\Omega \cap B_3(\Gamma_n)$ and is less than $2 + 3^{1/4} - 2^{1/4} \leq 3$ everywhere.  

Our goal is to bound the derivatives of $\theta_+^3$ so that we can apply a compactness argument to the sequence $\theta_n$.  (For the curious reader, we will point out the steps below in which it is important to consider $\theta_+^3$ instead of $\theta_+$.)  

Apply the energy inequality Lemma \ref{thm:energy inequality} to $\theta$ and $\psi(x-\Gamma_n)$, and find that for some $C$ independent of $n$
\begin{equation} \label{ddt theta bounded} \ddt \int \theta_+^2 \leq C \end{equation}
and moreover that
\begin{equation}\label{DG2 energy} \sup_{[-4,0]} \int \theta_+^2 + \int_{-4}^0 \int \abs{\Lambda^{1/2}\theta_+}^2 + \int_{-4}^0 \int \Lambda^{1/2}\theta_+ \Lambda^{1/2}\theta_- \leq C. \end{equation}
This proves in particular that $\theta_+ \in L^2(-4,0; \HD^{1/2}(\Omega))$ is uniformly bounded.  

What's more, $\norm{\theta_+^3}_{L^2(-4,0;\HD^{1/2}(\Omega_n))}$ is uniformly bounded because
\begin{align*} 
\norm{\Lambda^{1/2}(\theta_+^3)}_2^2 &= \iint [\theta_+(x)^3 - \theta_+(y)^3]^2 K + \int \theta_+^6 B 
\\ &\leq 2\iint \theta_+(x)^4 [\theta_+(x)-\theta_+(y)]^2 K + 2\iint \theta_+(y)^4[\theta_+(x)-\theta_+(y)]^2 K + \norm{\theta_+}_\infty^4 \int \theta_+^2 B
\\ &\leq C \norm{\theta_+}_\infty^4 \norm{\theta_+}_{\HD^{1/2}}^2.  
\end{align*}

By Lemma \ref{thm:hadamard 3 lines}, if $E \theta_+^3$ is the zero-extension of $\theta_+^3$ from $\Omega_n$ to $\R^2$, then
\begin{equation} \label{theta3 compact in space} \norm{ E \theta_+^3}_{L^2(-4,0; H^{1/2}(\R^2))} \leq C \end{equation}
where $C$ does not depend on $n$.  

Since $\theta_n$ solves the equation
\[ \del_t \theta_n + (\uhigh + \ulow)\cdot \grad \theta_n + \Lambda \theta_n = 0, \] 
multiply this equation by $\varphi \theta_+^2$, where $\varphi$ is any function in $C^2(\R^2)$ restricted to $\Omega_n$, and integrate to obtain
\begin{align*} 
\frac{1}{3} \int \varphi \del_t \theta_+^3 + \frac{1}{3} \int \varphi \dot{\Gamma}_n \cdot \grad \theta_+^3 &= \frac{-1}{3} \int \varphi (\ulow^n - \dot{\Gamma}_n + \uhigh^n) \cdot \grad \theta_+^3 - \int \varphi \theta_+^2 (\ulow^n - \dot{\Gamma}_n + \uhigh^n) \cdot \grad \psi 
\\ & - \int \varphi \theta_+^2 \Lambda \theta_+ - \int \varphi \theta_+^2 \Lambda \psi + \int \varphi \theta_+^2 \Lambda \theta_-. 
\end{align*}
Further rearranging, this becomes
\begin{align*} 
\int \varphi \del_t \theta_+^3 + \int \varphi \dot{\Gamma}_n \cdot \grad \theta_+^3 
&= \int (\ulow^n - \dot{\Gamma}_n) \cdot (\theta_+^3 \grad\varphi - 3 \varphi \theta_+^2 \grad\psi) + \int \Lambda^{-1/4} \uhigh^n \Lambda^{1/4} \paren{\theta_+^3 \grad\varphi - 3 \varphi \theta_+^2 \grad\psi}
\\ & - 3\int \varphi \theta_+^2 \Lambda \theta_+ - 3\int \varphi \theta_+^2 \Lambda \psi + 3\int \varphi \theta_+^2 \Lambda \theta_-. 
\end{align*}
We will bound the five terms on the right hand side one at a time.  

Each instance of $C$ in the following bounds is independent of $n$.  

\begin{itemize}
\item Consider the low-pass term.  As in the proof of Lemma~\ref{thm:energy inequality}, we have $|\ulow^n(t,x) - \dot{\Gamma}_n(t)| \leq (1+8\kappa)\Cgamma + 6 \kappa$ for $t \in [-4,0]$ and $x \in \supp(\theta_+) \subseteq \Omega_n \cap B_3(\Gamma_n(t))$.  Thus for $t \in [-4,0]$ we have for $C$ independent of $n$ and of $\varphi$
\[ \int (\ulow^n - \dot{\Gamma}_n) \cdot (\theta_+^3 \grad\varphi - 3 \varphi \theta_+^2 \grad\psi) \leq C \paren{\norm{\grad\varphi(t,\cdot)}_{L^\infty(\Omega)} + \norm{\varphi(t,\cdot)}_{L^\infty(\Omega)}}. \]

\item Consider the high-pass term.  By Lemma \ref{thm:Lambda stuff} parts \ref{thm:L1 of Lambda1/4 bounded} and \ref{thm:extra product rule}, 
\[ \int \Lambda^{-1/4} \uhigh^n \Lambda^{1/4} \paren{\theta_+^3 \grad\varphi - 3 \varphi \theta_+^2 \grad\psi} \leq C \kappa |\supp(\theta_+)|^{1/2} \paren{\norm{\theta_+^3 \grad\varphi}_{L^2} + \norm{\varphi \theta_+^2 \grad\psi}_{L^2} + \norm{\theta_+^3 \grad\varphi}_{\HD^{1/2}} + \norm{\varphi \theta_+^2 \grad\psi}_{\HD^{1/2}}}. \]
\[ \leq C \paren{ \norm{\varphi(t,\cdot)}_{C^1(\Omega)} + \norm{\varphi(t,\cdot)}_{C^2(\Omega)} \norm{\theta(t,\cdot)}_{\HD^{1/2}} }. \]

\item Consider the $\Lambda \theta_+$ term.  Decomposing this term using Proposition~\ref{thm:Caff Stinga representation} we have first an interior term $\iint [\varphi(x)\theta_+(x)^2 - \varphi(y)\theta_+(y)^2][\theta_+(x)-\theta_+(y)] K$ which decomposes as
\[ \iint \varphi(x)(\theta_+(x)+\theta_+(y))[\theta_+(x)-\theta_+(y)]^2 K + \iint \theta_+(y)^2 [\varphi(x)-\varphi(y)][\theta_+(x)-\theta_+(y)] K. \]
The first part is bounded by the $L^\infty$ norms of $\varphi$ and $\theta_+$ and the square of the $\HD^{1/2}$ norm of $\theta_+$, while the second part is bounded
\[ \iint \theta_+(y)^2 [\varphi(x)-\varphi(y)][\theta_+(x)-\theta_+(y)] K \leq \norm{\theta_+(t,\cdot)}_{\HD^{1/2}} \sqrt{\int \theta_+(y)^2 \int \frac{[\varphi(x)-\varphi(y)]^2}{|x-y|^3} dx \,dy}\]
which is bounded by the $C^1$ norm of $\varphi$ and the $\HD^{1/2}$ norm of $\theta_+$.  
% (using H\"{o}lder's inequality) by the $L^\infty$ and $\HD^{1/2}$ norms of $\theta_+$ and the square root of $\sup_y \int [\varphi(x)-\varphi(y)]^2/|x-y|^3 \,dx$.  This last quantity is in turn bounded by the $C^1$ norm of $\varphi$.  

The boundary term $\int \varphi \theta_+^3 B$ is bounded by the $L^\infty$ norms of $\varphi$ and $\theta_+$, and by $\int \theta_+^2 B$ which is less than $\norm{\theta_+(t,\cdot)}_{\HD^{1/2}}$.  Taken together we have
\[ \int \varphi \theta_+^2 \Lambda \theta_+ \leq C \paren{ \norm{\varphi(t,\cdot)}_{L^\infty(\Omega)} \norm{\theta_+(t,\cdot)}_{\HD^{1/2}}^2 + \norm{\varphi(t,\cdot)}_{C^1(\Omega)} \norm{\theta_+(t,\cdot)}_{\HD^{1/2}} }. \]

\item Consider the $\Lambda\theta_-$ term.  For any non-negative function $f$ we know by Lemma~\ref{thm:Lambda stuff} part \eqref{thm:disjoint} that $\int f \theta_+ \Lambda \theta_- \leq 0$.  
It follows that $-\theta_+ \Lambda \theta_-$ is a pointwise non-negative distribution.  Moreover, the integral over $[-4,0]\times\Omega$ of $-\theta_+\Lambda\theta_-$ is bounded by \eqref{DG2 energy}.  Thus $\theta_+\Lambda\theta_-$ is a measure with bounded total-variation norm.  In fact, because $\varphi$ is a continuous function,
\[ \int_{-4}^0 \int \varphi \theta_+^2 \Lambda \theta \leq \norm{\theta_+}_\infty \norm{\theta_+\Lambda\theta_-}_{\mathcal{M}} \norm{\varphi}_{C^0} \leq C \norm{\varphi}_{L^\infty([-4,0]\times\Omega)}. \]

\item Consider the $\Lambda \psi$ term.  Decomposing this term using Proposition~\ref{thm:Caff Stinga representation} we have first an interior term $\iint [\varphi(x) \theta_+(x)^2 - \varphi(y) \theta_+(y)^2][\psi(x)-\psi(y)] K$ which decomposes as
\[ \int \theta_+(y)^2\int[\varphi(x) - \varphi(y)][\psi(x)-\psi(y)] K + \iint \varphi(x) [\theta_+(x)^2 - \theta_+(y)^2][\psi(x)-\psi(y)] K. \]
The first part is bounded by the $C^1$ norms of $\varphi$ and $\psi$ and the $L^2$ norm of $\theta_+$, while the second part is bounded
\[ \iint \varphi(x) [\theta_+(x)^2 - \theta_+(y)^2][\psi(x)-\psi(y)] K \leq \norm{\theta_+^2(t,\cdot)}_{\HD^{1/2}} \sqrt{ \int \varphi(x)^2 \int \frac{[\psi(x)-\psi(y)]^2}{|x-y|^3}dy \,dx} \]
which is bounded, because $\psi$ is smooth and globally $1/4$-H\"{o}lder continuous, by $L^2$ norm of $\varphi$ and the $\HD^{1/2}$ norm of $\theta_+$.  

The boundary term $\int \varphi \theta_+^2 \psi B$ is bounded by the $L^\infty$ norms of $\varphi$ and $\psi \indic{\theta_+>0}$ and by $\int \theta_+^2 B$ which is less than $\norm{\theta_+(t,\cdot)}_{\HD^{1/2}}^2$.  Taken together we have
\[ \int \varphi \theta_+^2 \Lambda \psi \leq C \paren{ \norm{\varphi(t,\cdot)}_{C^1(\Omega)} + \norm{\varphi(t,\cdot)}_{L^2(\Omega)} \norm{\theta_+(t,\cdot)}_{\HD^{1/2}} + \norm{\varphi(t,\cdot)}_{L^\infty(\Omega)} \norm{\theta_+(t,\cdot)}_{\HD^{1/2}}^2 }. \]
%
% Lastly, 
%\begin{align*} 
%\int \varphi \theta_+^2 \Lambda \psi &= \iint [\varphi(x) \theta_+(x)^2 - \varphi(y) \theta_+(y)^2][\psi(x)-\psi(y)] K + \int \varphi \theta_+^2 \psi B
%\\ &= \iint \varphi(x) [\theta_+(x)^2 - \theta_+(y)^2][\psi(x)-\psi(y)] K + \iint \theta_+(y)^2[\varphi(x) - \varphi(y)][\psi(x)-\psi(y)] K + \int \varphi \theta_+^2 \psi B
%\\ &\leq C\paren{\iint \varphi(x)^2 [\psi(x)-\psi(y)]^2 K}^{1/2} + \paren{ \norm{\varphi}_\infty + \norm{\grad \varphi}_\infty } \int \theta_+(y)^2 \,dy + \int \varphi \theta_+^2 \psi B
%\\ &\leq C \norm{\varphi}_{L^2} + C \norm{\varphi}_{\infty} + C \norm{\grad \varphi}_\infty + \norm{\psi\indic{\theta_+>0}}_\infty \norm{\theta_+}_{\HD^{1/2}}^2 \norm{\varphi}_{\infty}
%\\ &= C \norm{\psi}_{L^\infty(-4,0; L^2(\Omega_n))} + C \norm{\varphi}_{L^\infty(-4,0; L^\infty(\Omega_n))} + C \norm{\grad\varphi}_{L^\infty(-4,0; L^\infty(\Omega_n))}.
%\end{align*}

\end{itemize}

\begin{remark}
We are attempting to bound $\del_t \theta_+^3$.  If we had attempted to bound $\del_t \theta_+^p$ instead, the final three terms above would have been problematic for $p=1$ and the very final term would have been problematic for $p=2$.  
\end{remark}

Combining all of these bounds, and using the fact that $\theta_+ \in L^2(-4,0; \HD^{1/2})$ uniformly, we conclude that there exists a constant $C$ independent of $n$ such that, for any $\varphi \in L^\infty(-4,0; C^2(\R^2)) \cap L^\infty(-4,0; L^2(\R^2))$, 
\begin{equation} \label{theta3 compact in time} \int_{-4}^0 \int_{\Omega_n} \paren{ \del_t \theta_+^3 + \dot{\Gamma}_n \cdot \grad \theta_+^3 } \varphi \,dxdt \leq C \norm{\varphi}_{L^\infty(-4,0; C^2(\R^2))} + C \norm{\varphi}_{L^\infty(-4,0; L^2(\R^2))}. \end{equation}

Over time, the support of $\theta_+^3$ moves around in $\Omega_n$ following the path $\Gamma_n$.  In order to take a meaningful limit in $n$, we must shift these functions so that their supports remain in a compact set.  To that end, define a new function on $[-4,0] \times \R^2$ by
\[ v_n(t,x) := \begin{cases}
\theta_+(t, x + \Gamma_n(t))^3, & x + \Gamma_n(t) \in \Omega_n, \\
0, & x + \Gamma_n(t) \notin \Omega_n.
\end{cases} \]
In other words,
\begin{equation} \label{definition of v_n} v_n(t,x) = \paren{\theta_n(t, x + \Gamma_n(t)) - \psi(x)}_+^3 \end{equation}
when the right hand side is defined.

Let $X \subseteq C^2(\R^2)$ be the Banach space of $C^2$ functions with norm $\norm{\cdot}_X = \norm{\cdot}_{C^2(\R^2)} + \norm{\cdot}_{L^2(\R^2)}$ finite.  Note that
\[ \del_t v_n(t,x) = \del_t \theta_+^3(t,x+\Gamma_n) + \dot{\Gamma}_n \cdot \grad \theta_+^3(t,x+\Gamma_n). \]

We know from \eqref{theta3 compact in space} that
\[ \norm{ v_n }_{L^2(-4,0; H^{1/2}(\R^2)} \leq C \]
and from \eqref{theta3 compact in time} that
\[ \norm{ \del_t v_n }_{L^1(-4,0; X^*)} \leq C. \]

According to the Aubin-Lions Lemma, the set $\{v_n\}_n$ is therefore compactly embedded in $L^2([-4,0]\times\R^2)$.  Up to a subsequence, there is a function $v \in L^2([-4,0]\times\R^2)$ such that
\[ v_n \to v. \]
By elementary properties of $L^2$ convergence, we know that $v \in L^\infty$, $\supp(v) \subseteq [-4,0]\times B_3(0)$, and $v \in L^2(H^{1/2})$.  

By \eqref{ddt theta bounded}
\begin{equation} \label{ddt v bounded} \ddt \int_{\R^2} v_n^{2/3} \,dx = \ddt \int_{\Omega_n} \theta_+^2 \,dx \leq C \end{equation}
so the same must be true of $v$.

By \eqref{mass assumption above}, \eqref{mass assumption between}, and \eqref{mass assumption below} applied to $v_n$ (recalling the relation \eqref{definition of v_n}), we conclude that
\begin{equation} \label{mass assumptions v} \begin{cases} 
\abs{\{v \geq 1\} \cap [-2,0]\times B_2(0)} &\geq \delta_0/4, \\
\abs{\{0 < v < [1-\psi]^3\} \cap [-4,0]\times B_4(0)} &\leq 0, \\
\abs{\{v \leq 0\} \cap [-4,0]\times B_4(0)} &\geq 2|B_4|
\end{cases} \end{equation}

%\begin{align} 
%\abs{\{v_n \geq 1\} \cap [-2,0]\times B_2(0)} &\geq \delta_0, \\
%\abs{\{0 < v_n < [1-\psi(x)]^3\} \cap [-4,0]\times B_4(0)} &\leq 1/n, \\
%\abs{\{v_n \leq 0\} \cap [-4,0]\times B_4(0)} &\geq 2|B_4|
%\end{align}

For any $(t,x) \in [-4,0]\times B_4(0)$, either $v(t,x) \geq [1 - \psi(x)]^3$ or else $v(t,x) = 0$.  In fact, since $\norm{v(t,\cdot)}_{H^{1/2}} < \infty$ for almost every $t$ and $H^{1/2}$ does not contain functions with jump discontinuities, the function $v$ is either identically 0 or else $\geq [1-\psi(x)]^3$ at each $t$.  

Thus $\int v(t,x)^{2/3} \,dx$ is either 0 or else $\geq \int [1-\psi(x)]^3 \,dx > 0$ at each $t$.  By \eqref{ddt v bounded} and \eqref{mass assumptions v}, $v$ must be identically zero for all $t > -2$ but also must be non-zero for some $t > -2$, which is a contradiction.  Our assumption that the sequence $\theta_n$ exists must have been false.  The proposition must be true.  

\end{proof}


%-------%-------%-------%-------%-------%-------%-------%-------%-------%-------%-------%-------

\section{A Decrease in Oscillation} \label{sec:harnack}

We put together Propositions \ref{thm:DG1} and \ref{thm:DG2} to produce an Oscillation lemma.  It allows us to improve a bound on the supremum of a function based on information about its positive support.  

\begin{proposition}[Oscillation Lemma] \label{thm:oscillation general}
Let $\kappa$, $\Comega$, and $\Cgamma$, be positive constants. Then there exists a constant $k_0>0$ such that the following holds:

Let $\Omega \subseteq \R^2$ be a bounded open set with $C^{2,\beta}$ boundary for some $\beta \in (0,1)$.  Assume that Lemma~\ref{thm:Caff Stinga representation} hold on $\Omega$ with kernels that satisfy
\[ K_{1/4}(x,y) \leq \Comega |x-y|^{1/2} K_{1}. \]

Let $\theta$, $\ulow$, $\uhigh$, $\Gamma$ and $\gamma$ solve \eqref{eq:main linear brokendown} on $[-2,0]\times\Omega$, and satisfy $\norm{\Lambda^{-1/4} \uhigh}_{L^\infty([-2,0]\times\Omega)} \leq 6 \kappa$, $\norm{\grad \ulow}_{L^\infty([-2,0]\times\Omega)} \leq 2\kappa$, and $\norm{\dot{\gamma}}_{L^\infty([-2,0])} \leq \Cgamma$.  


Suppose that for all $t \in [-5,0]$ and any $x \in \Omega$
\[ \theta(t,x) \leq 2 + 2^{-k_0} \paren{|x-\Gamma(t)|^{1/4}-2^{1/4}}_+, \]
and that
\[ \abs{\{\theta \leq 0\} \cap [-4,0]\times B_4(\Gamma)} \geq \frac{4|B_4|}{2}. \]

Then for all $t \in [-1,0]$, $x \in \Omega \cap B_1(\Gamma)$ we have
\[ \theta(t,x) \leq 2 - 2^{-k_0}. \]
\end{proposition}

\begin{proof}
Let $\mu$ and $\delta_0$ as in Proposition \ref{thm:DG2}, and take $k_0$ large enough that $(k_0-1) \mu > 4 |B_4|$.  

Consider the sequence of functions,
\[ \theta_k(t,x) := 2 + 2^k (\theta(t,x) - 2). \]
That is, $\theta_0 = \theta$ and as $k$ increases, we scale vertically by a factor of 2 while keeping height 2 as a fixed point.  Note that since $\theta$ satisfies [cite, boundedness], each $\theta_k$ for $k \leq k_0$ and $(t,x) \in [-5,0] \times \Omega$ satisfies
\[ \theta_k(t,x) \leq 2 + \paren{|x-\Gamma(t)|^{1/4}-2^{1/4}}_+. \]
This is precisely the assumption in Proposition \ref{thm:DG2}.  

Note also that
\[ \abs{\{\theta_k \leq 0\} \cap [-4,0]\times B_4(\Gamma)} \]
is an increasing function of $k$, and hence is greater than $2|B_4|$ for all $k$.  

Assume, for means of contradiction, that
\[ \abs{\{1 \leq \theta_k \} \cap [-2,0]\times B_2(\Gamma)} \geq \delta_0/4 \]
for $k = k_0-1$.  Since this quantity is decreasing in $k$, it must then exceed $\delta_0/4$ for all $ k < k_0$ as well.  

Applying Proposition \ref{thm:DG2} to each $\theta_k$, we conclude that 
\[ \abs{\{0 < \theta_k < 1\} \cap [-4,0]\times B_4(\Gamma)} \geq \mu. \]
In particular, this means that the quantity [cite] increases by atleast $\mu$ every time $k$ increases by 1. By choice of $k_0$ and the fact that quantity [cite] is bounded by $4|B_4|$, we obtain a contradiciton.  Therefore, the assumption [cite] must fail for $k = k_0-1$.  

Therefore $\theta_{k_0}$ must satisfy the assumptions of Proposition \ref{thm:DG1}.  In particular, we conclude that
\[ \theta_{k_0}(t,x) \leq 1 \qquad \forall t \in [-1,0], x \in \Omega \cap B_1(\Gamma). \]

For the original function $\theta$, this means that
\[ \theta(t,x) \leq 2 - 2^{-k_0} \qquad \forall t \in [-1,0], x \in \Omega \cap B_1(\Gamma). \]
\end{proof}

That's the absolute gain.  Now let us consider how this gain can be shifted to our new reference frame.  But first, a quick technical lemma:

\begin{lemma} \label{thm:technical scaling of barrier}
There exist constants $\bar{\lambda} > 0$ and $\alpha > 1$ such that, for any $0 < \eps \leq 1/2$ and any $z \geq 1$
\[ \paren{|\eps\n (z - 1) + 3|^{1/4} - 2^{1/4}}_+ - \alpha \paren{|z|^{1/4} - 2^{1/4}}_+ \geq \bar{\lambda}. \]
\end{lemma}

\begin{proof}
For $z$ fixed, this function is increasing as $\eps$ decreases, so it will suffice to show the lemma when $\eps = 1/2$.  Consider
\[ \paren{|2 z + 1|^{1/4} - 2^{1/4}}_+ - \alpha \paren{|z|^{1/4} - 2^{1/4}}_+. \]
When $\alpha = 1$, this quantity is clearly non-negative and in fact strictly positive when $z \geq 1$.  On any compact interval $[0,N]$, the quantity with $\alpha = 1$ is bounded below, and the quantity $\paren{|z|^{1/4} - 2^{1/4}}_+$ is bounded above, so if $\alpha-1$ is less than the ratio of those bounds then the total quantity will be bounded below.  

However, the range of acceptable $\alpha$ depends on $N$, and it is possible that no single $\alpha$ is acceptable for the whole of $z \in [1,\infty)$.  

For $z > 2$, the expression reduces to
\[ (2z+1)^{1/4} - \alpha z^{1/4} - (\alpha-1) 2^{1/4} = z^{1/4} \paren{(2 + 1/z)^{1/4} - \alpha} - (\alpha-1)2^{1/4}. \]
This quantity is increasing as $\alpha$ decreases, and for any $\alpha < 2^{1/4}$ it tends to $\infty$ as $z$ increases. 

This is sufficient to show that for some $\alpha > 1$, there exists a lower bound $\bar{\lambda}$ on the quantity [cite], and thus the lemma holds. 
\end{proof}

We are ready to prove the shifted version of the Harnack Inequality.  

\begin{lemma}[Oscillation Lemma, with shift] \label{thm:oscillation shifted}
Let $\kappa$, $\Comega$, and $\Cgamma$, be positive constants, and let $k_0$ be as in Lemma~\ref{thm:oscillation general}.  Then there exists a constant $\lambda > 0$ such that the following holds:

Let $\Omega \subseteq \R^2$ be a bounded open set with $C^{2,\beta}$ boundary for some $\beta \in (0,1)$.  Assume that Lemma~\ref{thm:Caff Stinga representation} hold on $\Omega$ with kernels that satisfy
\[ K_{1/4}(x,y) \leq \Comega |x-y|^{1/2} K_{1}. \]

Let $\theta$, $\ulow$, $\uhigh$, $\Gamma$ and $\gamma$ solve \eqref{eq:main linear brokendown} on $[-5,0]\times\Omega$, and satisfy $\norm{\Lambda^{-1/4} \uhigh}_{L^\infty([-5,0]\times\Omega)} \leq 6 \kappa$, $\norm{\grad \ulow}_{L^\infty([-5,0]\times\Omega)} \leq 2\kappa$, and $\norm{\dot{\gamma}}_{L^\infty([-5,0])} \leq \Cgamma$.  

Suppose that for all $t \in [-5,0]$ and any $x \in \Omega$
\begin{equation} \label{theta bounded everywhere} |\theta(t,x)| \leq 2 + 2^{-k_0} \paren{|x-\Gamma(t)|^{1/4}-2^{1/4}}_+ \end{equation}
and that
\[ \abs{\{\theta \leq 0\} \cap [-4,0]\times B_4(\Gamma)} \geq 2|B_4|. \]

Then for any $\eps \in (0,1/5)$ such that
\begin{equation} \label{Cgamma and eps for harnack} 5 \Cgamma \leq \eps\n - 3 \end{equation}
we have
\[ \abs{\frac{2}{2-\lambda} \bracket{\theta(\eps t, \eps x) + \lambda}} \leq 2 + 2^{-k_0} \paren{|x-\eps\n\Gamma(\eps t)-\eps\n\gamma(\eps t)|^{1/4}-2^{1/4}}_+. \]
for all $t \in [-5,0]$ and $x$ such that $\eps x \in \Omega$.  
\end{lemma}

If we only wish to show that by zooming horizontally by a large amount and zooming and translating vertically by a small amount we stay under the barrier, this is obvious and merely requires being written down.  Even the shift itself is clearly not a problem when considered in the un-zoomed coordinates.  Since the velocity of $\gamma$ is bounded by $\Cgamma$, the shift $\gamma$ is arbitrarily small over very small time periods.  The important thing to pay attention for is the dependence of $\eps$ and $\Cgamma$ and $k_0$ on eachother.  

As we will see in Section~\ref{sec:holder} when we apply this lemma, the constant $\Cgamma$ depends on $\eps$ and $k_0$ depends on $\Cgamma$.  In the following proof, the constant $\eps$ will need to be small relative to $\Cgamma$.  The assumption \eqref{Cgamma and eps for harnack} in this lemma turns out to be satisfiable, and now we must prove that it is sufficient.  

%from the fact that $k_0$ itself depends on $\Cgamma$, and as we will see in the next section $\Cgamma$ depends on $\eps$, so $\eps$ cannot depend on $k_0$.  In time, the co-dependence of $\eps$ and $\Cgamma$ is easy to untangle (so long as $\eps \Cgamma$ is less than some universal constant, the proof will go through).  In space, it is less clear that $\eps$ will not depend on $k_0$, and of course we need to zoom in time and space by the same factor so both issues are interconnected.  

\begin{proof}
Take $\lambda$ such that
\begin{equation} \label{lambda smallness assumptions} 2\lambda \leq 2^{-k_0}, \qquad (2+\lambda)(\frac{2}{2-\lambda}) \leq 2 + 2^{-k_0} \bar{\lambda}, \qquad \frac{2}{2-\lambda} \leq \alpha. \end{equation}
for $\bar{\lambda}$ and $\alpha$ from Lemma \ref{thm:technical scaling of barrier}.  

Denote 
\[ \bar{\theta}(t,x) := \frac{2}{2-\lambda} \bracket{\theta(\eps t, \eps x) + \lambda} \]
defined for $t \in [-5/\eps,0]$ and 
\[ x \in \Omega_\eps := \{x\in \R^2: \eps x \in \Omega\} \]
and denote
\[ \phi(x) := \paren{|x|^{1/4} - 2^{1/4}}_+. \]

We already proved in Lemma \ref{thm:oscillation general} that $\theta \leq 2 - 2^{-k_0}$ for $t \in [-1,0]$ and $x \in \Omega\cap B_1(\Gamma)$.  On this same set, $\theta \geq -2$ by assumption.  For $\bar{\theta}$, this means that when $t \in [-1/\eps, 0]$ and $x \in \Omega \cap B_{1/\eps}(\eps\n \Gamma(\eps t))$,
\begin{equation}\label{bar theta bounded basin} \begin{cases}
\bar{\theta}(t,x) &\leq \frac{2}{2-\lambda} \bracket{2-2^{-k_0}+\lambda} \leq \frac{2}{2-\lambda} \bracket{2-\lambda} = 2. \\
\bar{\theta}(t,x) &\geq \frac{2}{2-\lambda} \bracket{-2+\lambda} = -2.
\end{cases} \end{equation}

Similarly, the bound \eqref{theta bounded everywhere} on $\theta$ becomes the equivalent bounds on $\bar{\theta}$, for all $(t,x) \in [-5/\eps,0] \times \Omega_\eps$
\begin{equation} \bar{\theta}(t,x) \leq \frac{2}{2-\lambda} \bracket{2 + \lambda + 2^{-k_0} \phi(|\eps x - \Gamma(\eps t)|)} \label{bar theta bounded above everywhere} \end{equation}
and
\begin{equation} \bar{\theta}(t,x) \geq \frac{2}{2-\lambda} \bracket{- 2 + \lambda - 2^{-k_0} \phi(|\eps x - \Gamma(\eps t)|)}. \label{bar theta bounded below everywhere} \end{equation}

It remains to show that these bounds \eqref{bar theta bounded basin}, \eqref{bar theta bounded above everywhere}, and \eqref{bar theta bounded below everywhere} on $\bar{\theta}$ imply the bound stipulated by the proposition.  

Let $t \in [-5,0]$ and $x \in \Omega_\eps$, and define 
\[ y := x - \eps\n \Gamma(\eps t). \]

From \eqref{bar theta bounded above everywhere} and the assumptions \eqref{lambda smallness assumptions}, we can bound
\begin{align*} 
\bar{\theta}(t,x) &\leq \frac{2}{2-\lambda} \bracket{2 + \lambda + 2^{-k_0} \phi(\eps |y|)}
\\ &\leq 2 + 2^{-k_0} \bar{\lambda} + 2^{-k_0} \alpha \phi(\eps |y|)
\\ &= 2 + 2^{-k_0} \bracket{\bar{\lambda} + \alpha \phi(\eps |y|)}.
\end{align*}

From \eqref{bar theta bounded below everywhere} and the assumptions \eqref{lambda smallness assumptions}, we can bound
\begin{align*}
-\bar{\theta}(t,x) &\leq \frac{2}{2-\lambda} \bracket{2 -\lambda + 2^{-k_0} \phi(\eps |y|)}
\\ &\leq 2 + 2^{-k_0} \alpha \phi(\eps |y|)
\\ &\leq 2 + 2^{-k_0} \bracket{\bar{\lambda} + \alpha \phi(\eps |y|)}.
\end{align*}

Therefore
\begin{equation} \label{bar theta bounded everywhere} \abs{\bar{\theta}(t,x)} \leq 2 + 2^{-k_0} \bracket{\bar{\lambda} + \alpha \phi(\eps |y|)}. \end{equation}

If $|y| \leq \eps\n$ then from \eqref{bar theta bounded basin} we have
\[ \abs{\bar{\theta}(t,x)} \leq 2 \leq 2 + 2^{-k_0} \phi(x - \eps\n \Gamma(\eps t) - \eps\n \gamma(\eps t)) \]
and the proof would be complete.  Therefore assume without loss of generality that $|y|\geq \eps\n$.  In this case we can apply Lemma \ref{thm:technical scaling of barrier} so
\[ 2 + 2^{-k_0} \bracket{\bar{\lambda} + \alpha \phi(\eps |y|)} \leq 2 + 2^{-k_0} \phi(|y|-\eps\n + 3). \]

For $t \in [-5,0]$, we have by assumption \eqref{Cgamma and eps for harnack}
\[ |y|-\eps\n+3 \leq |y| - 5 \Cgamma \leq |y - \eps\n\gamma(\eps t)|. \]

The estimate \eqref{bar theta bounded everywhere} becomes
\[ \abs{\bar{\theta}(t,x)} \leq 2 + 2^{-k_0} \phi(|x - \eps\n\Gamma(\eps t) - \eps\n\gamma(\eps t)|). \]

This concludes the proof.  
\end{proof}

%-------%-------%-------%-------%-------%-------%-------%-------%-------%-------%-------%-------

\section{H\"{o}lder Continuity} \label{sec:holder}

In this section we shall prove the main theorem, Theorem~\ref{thm:main continuity}.  We begin with a final lemma to describe the scaling properties of \eqref{eq:main linear}.  

\begin{lemma}[Scaling] \label{thm:scaling}
Let $\Omega \subseteq \R^2$ be a bounded set with $C^{2,\alpha}$ boundary.  Suppose that $\theta:[-T,0] \times \Omega \to \R$ and $u:[-T,0]\times \Omega \to \R^2$ solve \eqref{eq:main linear} and $u$ satisfies
\[ u = \sum_{j=j_0}^\infty u_j \]
with that sum converging in $L^2(\Omega)$ and $(u_j)_j$ calibrated with constant $\kappa$ and center $N$.  Suppose that on $\Omega$ the functions $K_{1/4}$ and $K_1$ (defined in Proposition~\ref{thm:Caff Stinga representation}) satisfy the relation
\begin{equation} \label{scaling 1/4 to 1 property} K_{1/4}(x,y) \leq \Comega |x-y|^{3/4} K_1(x,y) \qquad \forall x\neq y \in \Omega. \end{equation}

Let $\eps>0$ be a small constant. 

Then
\[ \bar{\theta}(t,x) := \theta(\eps t, \eps x) \]
and
\[ \bar{u}(t,x) := \sum_{j=j_0}^\infty u_j(\eps t, \eps x) \]
satisfies the same PDE on $[-T/\eps, 0]\times \Omega_\eps$ where $\Omega_\eps = \{x \in \R^2: \eps x \in \Omega\}$.  

Moreover, $(u_j)_j$ is calibrated with the same constant $\kappa$ but with center $N - \ln_2(\eps)$, and the estimate
\[ \bar{K}_{1/4}(x,y) \leq \Comega |x-y|^{3/4} \bar{K}_1(x,y) \qquad \forall x\neq y \in \Omega_\eps \]
holds.  

%\[ \norm{\sum_{j=-\infty}^0 \bar{u}_j(t,0) - \sum_{j=-\infty}^0 u_j(\eps t, 0)}_\infty \leq \kappa N. \]
\end{lemma}

\begin{proof}
Denote by $\bar{\Lambda}$ the square root of the Laplacian with Dirichlet boundary conditions on $\Omega_\eps$.  One can calculate (see e.g. \cite{CaSt} Section 2.4) that for $(t,x) \in [-T/\eps, 0]\times \Omega_\eps$
\[ \Lambda \theta(\eps t,\eps x) = \eps \bar{\Lambda} \bar{\theta}(t,x). \]

Similarly, in the Caffarelli-Stinga representation from Proposition~\ref{thm:Caff Stinga representation} the operator $\bar{\Lambda}^s$ will have kernel
\[ \bar{K}_s(x,y) = \eps^s K_s(\eps x,\eps y). \]

From these facts it is clear that the scaled functions satisfy \eqref{eq:main linear} and \eqref{scaling 1/4 to 1 property}.  

%We calculate
%\[ \del_t \bar{\theta}(p) = \eps \del_t \theta(\bar{p}) \]
%and 
%\[ \grad \bar{\theta}(p) = \eps \grad \theta(\bar{p}) \]
%and
%\[ \Lambda \bar{\theta}(p) = \eps \Lambda \theta(\bar{p}). \]
%
%\& cetera...

Define
\[ \bar{u}_j(t,x) := u_j(\eps t, \eps x). \]
To show that $(\bar{u}_j)_{j \in \Z}$ is calibrated, we must translate the various bounds on $u_j$ to corresponding bounds on $\bar{u}_j$.  Each of the calculations are similar, so we show only one:
\[ \norm{\grad \bar{u}_j}_\infty = \eps \norm{\grad u_j}_\infty \leq 2^{\ln_2(\eps)} 2^j 2^{-N} \kappa = 2^j 2^{-(N-\ln_2(\eps))} \kappa. \]

\end{proof}


\begin{proof}[Proof of Theorem \ref{thm:main continuity}]
We'll show that if $\theta$ with $\norm{\theta}_{L^\infty([-5,0] \times \Omega)} \leq 2$ solves \eqref{eq:main nonlinear} on $[-5,0]\times \Omega$ then $\theta$ is H\"{o}lder continuous at the point $(0,0)$ (with possibly $0 \in \bar{\Omega}$).  Up to translation and scaling, this will be sufficient to show continuity at all points in the domain, with a constant depending on $\Omega$ and on the time we wait.  

From Section \ref{sec:littlewood paley}, we know that 
\[ \R^\perp \theta = \sum_{j=j_0}^\infty u_j \]
for a sequence $(u_j)_j$ calibrated with some constant $\kappa = \kappa(\Omega)$ and center 0.  

Choose a constant $0 < \eps < 1/5$ such that
\begin{equation}\label{eps is small enough for Cgamma} 
5 \max\paren{ - \kappa \ln_2(\eps) e^{10\eps\kappa}, (1-j_0) \kappa} \leq \eps\n - 3,
\end{equation}

%\[ 5 \max\paren{ - \kappa \ln_2(\eps) e^{10\eps\kappa}, (1-j_0) \kappa} \leq \eps^{-a k}/k \qquad \forall k\geq 0. \]
%critical point at $k = -1/(a \ln(\eps))$, at that point the bound is $\leq -a \ln(\eps) \eps^{1/\ln(\eps)} = - a \ln(\eps) e$

For notational convenience, denote
\[ \sum_k = \sum_{j > - k \ln(\eps)}, \qquad \sum^k = \sum_{j \leq -k \ln(\eps)}. \]

For integers $k \geq 0$ consider the domains
\[ \Omega_k := \{x \in \R^2: \eps^k x \in \Omega\} \]
and define the following functions on $[-5,0]\times \Omega_k$:
\begin{align*} 
\ulowth{k}(t,x) &:= \sum^k u_j(\eps^k t, \eps^k x), \\
\uhighth{k}(t,x) &:= \sum_k u_j(\eps^k t, \eps^k x).  
\end{align*}

For $t \in [-5,0]$ and $k \geq 0$ define $\Gamma_k, \gamma_k: [-5,0] \to \R^2$ by the following ODEs:
\begin{align*}
\Gamma_0(t) &:= 0, \\
\gamma_k(0) &:= 0, \\
\dot{\gamma}_k(t) &:= \ulowth{k}(t, \Gamma_k(t) + \gamma_k(t)) - \dot{\Gamma}_k(t) \\
\Gamma_k(t) &:= \eps\n \gamma_{k-1}(\eps t) + \eps^{-2} \gamma_{k-2}(\eps^2 t) + \cdots + \eps^{-k} \gamma_0(\eps^k t), \qquad k \geq 1.
\end{align*}
Use [citation] some lemma from Bahouri-Chemin-Danchin that's a generalization of Picard-Lindelof to prove that these $\gamma$ exist.  Each $\ulowth{k}$ is a Lipschitz-in-space vector field, and each $\Gamma_k + \gamma_k$ is a flow along that vector field which ends up at the origin at $t=0$.  In particular, since  $\ulowth{k}$ is tangential to the boundary of $\Omega_k$ and has unique flows, the flow $\Gamma_k + \gamma_k$ cannot exit the region $\Omega_k$ and so our expressions remain well-defined.  

By Lemmas \ref{thm:scaling} and \ref{thm:calibration is good}, we know the sequence $(u_j(\eps^k \cdot, \eps^k \cdot))_j$ is calibrated and hence that independently of $k$
\[ \norm{\Lambda^{-1/4} \uhighth{k}}_{L^\infty([-5,0]\times \Omega_k)} \leq C \kappa \]
etc.  
Particularly
\[ \norm{\grad \ulowth{k}}_{L^\infty([-5,0]\times \Omega_k)} \leq 2 \kappa. \]

%I want to claim that 
%\[ \dot{\Gamma}_k(t) = \sum^{k-1} (\eps^k t, \eps^k \Gamma_k(t)). \]

%Notice that $\dot{\gamma}_0(t) = \ulowth{0}(t,\Gamma_0(t) + \gamma_0(t))$.  
By construction, for $k \geq 0$ we have $\Gamma_{k+1}(t) = \eps\n \gamma_k(\eps t) + \eps\n \Gamma_k(\eps t)$.  Therefore
\begin{align*} 
\dot{\Gamma}_{k+1}(t) &= \del_t \bracket{\eps\n \gamma_k(\eps t) + \eps\n \Gamma_k(\eps t)}
\\ &= \dot{\gamma}_k(\eps t) + \dot{\Gamma}_k(\eps t)
\\ &= \ulowth{k}(\eps t, \gamma_k(\eps t) + \Gamma_k(\eps t))
\\ &= \ulowth{k}(\eps t, \eps \Gamma_{k+1}(t)).  
\end{align*}

%Well, $\dot{\gamma}_0(t) = \ulowth{0}(t,\gamma_0(t))$.  Moreover, 
%\[ \ulowth{k}(t,\Gamma_{k+1}(t)) = \ulowth{k}(\eps\n \eps t, \eps\n \bracket{\gamma_k(\eps t) + \Gamma_k(\eps t)}) \]
%or
%\begin{align*} 
%\sum^k u_j(\eps^{k+1} t, \eps^{k+1} \Gamma_{k+1}(t)) &= \sum^k u_j(\eps^k \eps t, \eps^k \bracket{\gamma_k(\eps t) + \Gamma_k(\eps t)})
%\\ &= \ulowth{k}(\eps t, \gamma_k(\eps t) + \Gamma_k(\eps t))
%\\ &= \dot{\gamma}_k(\eps t) + \dot{\Gamma}_k(\eps t)
%\\ &= \del_t \bracket{\eps\n \gamma_k(\eps t) + \eps\n \Gamma_k(\eps t)}
%\\ &= \dot{\Gamma}_{k+1}(t).
%\end{align*}
%
%In other words,
%\[ \dot{\Gamma}_k(t) = \sum^{k-1} u_j(\eps^k t, \eps^k \Gamma_k(t)) \qquad k \geq 2. \]

With this in hand, we can bound the size of $\gamma_k$.  Namely, for $k \geq 1$, 
\begin{align*}
\dot{\gamma}_k(t) &= \ulowth{k}(t, \Gamma_k(t) + \gamma_k(t)) - \dot{\Gamma}_k(t)
\\ &= \ulowth{k}(t, \Gamma_k(t) + \gamma_k(t)) - \ulowth{k-1}(\eps t, \eps \Gamma_k(t))
\\ &= \sum^k u_j(\eps^k t, \eps^k \Gamma_k(t) + \eps^k \gamma_k(t)) - \sum^{k-1} u_j(\eps^k t, \eps^k \Gamma_k(t))
\\ &= \sum^{k-1} \bracket{u_j(\eps^k t, \eps^k \Gamma_k(t)+\eps^k \gamma_k(t)) - u_j(\eps^k t, \eps^k \Gamma_k(t))} + \sum_{k-1}^k u_j(\eps^k t, \eps^k \ldots)
\\ &= \bracket{\ulowth{k-1}\big(\eps t, \eps \Gamma_k(t)+\eps \gamma_k(t)\big) - \ulowth{k-1}(\eps t, \eps \Gamma_k(t))} + \sum_{k-1}^k u_j(\eps^k t, \eps^k \ldots).
\end{align*}
The sum $\sum^{k-1} u_j(\eps^k \cdot, \eps^k \cdot) = \ulowth{k-1}(\eps \, \cdot, \eps \, \cdot)$ is Lipschitz in space, with Lipschitz constant less than $2 \eps \kappa$.  Moreover, each $u_j$ has $\norm{u_j}_\infty \leq \kappa$.  Thus both terms of $\dot{\gamma}_k(t)$ are bounded
\[ |\dot{\gamma}_k(t)| \leq 2 \eps \kappa |\gamma_k(t)| - \kappa \ln_2(\eps). \]
This, by Gronwall's inequality, tells us that for $t \in [-5,0]$,
\[ |\gamma_k(t)| \leq \frac{-\ln_2(\eps)}{2 \eps} \paren{ e^{10 \eps \kappa} - 1} \]
and hence
\[ |\dot{\gamma}_k(t)| \leq -\kappa \ln_2(\eps) e^{10\eps \kappa}. \]

To account for $\gamma_0$, define
\[ \Cgamma = \max\paren{ - \kappa \ln_2(\eps) e^{10\eps\kappa}, j_0 \kappa} \]
so that for all $k \geq 0$ and $t \in [-5,0]$
\[ |\dot{\gamma}_k(t)| \leq \Cgamma. \]

Let us now produce a sequence of solutions $\theta_k$.  Define
\[ \theta_0(t,x) := \theta(t,x) \]
and for each $k \geq 0$, if $|\{\theta_k \leq 0\} \cap [-5,0]\times B_4(\Gamma_k(t))| \geq 2|B_4|$ then set
\[ \theta_{k+1}(t,x) := \frac{2}{2-\lambda} \bracket{\theta_k(\eps t, \eps x) + \lambda}. \]
Otherwise, set
\[ \theta_{k+1}(t,x) := \frac{1}{1-\lambda} \bracket{\theta_k(\eps t, \eps x) - \lambda}. \]

From Lemma \ref{thm:scaling}, we know that $\theta_k$ and the calibrated sequence $(u_j(\eps^k \cdot, \eps^k \cdot))_j$ solve \eqref{eq:main linear}.  

We will now show that
\begin{equation}\label{thetak below the barrier}
|\theta_k| \leq 2 + 2^{-k_0} \paren{|x-\Gamma_k(t)|^{1/4} - 2^{1/4}}_+
\end{equation}
holds for all $k \geq 0$.  

Since $|\theta_0|\leq 2$ by assumption, we know in particular that \eqref{thetak below the barrier} holds at $k=0$.  

This is sufficient for us to apply Lemma \ref{thm:oscillation shifted} to each $\theta_k$ (or to $-\theta_k$ as appropriate) in order.  We conclude that \eqref{thetak below the barrier} holds for all $k \geq 0$.  
%\[ \abs{\theta_{k+1}(t,x)} \leq 2 + 2^{-k_0} \paren{|x-\Gamma_{k+1}(t)|^{1/4} - 2^{1/4}}_+. \]

Each $\theta_k$ is between $-2$ and 2 on $[-5,0]\times B_2(\Gamma_k)$.  But recall that each $\Gamma_k$ is Lipschitz with constant $k \Cgamma$.  Thus $|\Gamma_k(t)|\leq 1$ for $t \in [-(k \Cgamma)\n, 0]$.  On that time interval, 
\[ \abs{\theta_k(t,x)} \leq 2 \qquad \forall x \in B_1(0). \]

We conclude that
\[ \abs{ \sup_{[-\eps^k (k \Cgamma)\n, 0] \times B_{\eps^k}(0)} \theta(t,x) - \inf_{[-\eps^k (k \Cgamma)\n, 0] \times B_{\eps^k}(0)} \theta(t,x) } \leq 4 \paren{\frac{2}{2-\lambda}}^{-k}. \]

In particular, for some positive constant $C$ such that
\[ \eps^{C k} \leq (k \Cgamma)\n \qquad \forall k \geq 0, \]
we can say that
\[ |t|^2 + |x|^2 \leq \eps^{(1+C)k} \]	
implies that $(t,x) \in [-\eps^k (k \Cgamma)\n, 0] \times B_{\eps^k}(0)$ which in turn implies that
\[ \abs{\theta(t,x) - \theta(0,0)} \leq  4 \paren{\frac{2}{2-\lambda}}^{-k}. \]

In other words,
\begin{align*} 
\abs{\theta(t,x) - \theta(0,0)} &\leq 4 \paren{\frac{2}{2-\lambda}}^{ -\frac{1}{1+C} \log_\eps(|t|^2 - |x|^2)  + 1} 
\\ &= 4 \paren{\frac{2}{2-\lambda}} \exp\bracket{\ln\paren{\frac{2}{2-\lambda}} \frac{\ln(|t|^2 + |x|^2)}{-(1+C)\ln(\eps)}}
\\ &= \frac{8}{2-\lambda} (|t|^2 + |x|^2)^{-\frac{\ln(2) - \ln(2-\lambda)}{(1+C)\ln(\eps)}}.
\end{align*}

\end{proof}

%-------%-------%-------%-------%-------%-------%-------%-------%-------%-------%-------%-------


\appendix
\section{Proof of H\"{o}lder Interpolation}

In this appendix we prove the lemma \ref{thm:Holder interpolation}.  

\begin{proof}[Proof of \ref{thm:Holder interpolation}]
The first claim is straigtforward.  
\begin{align*} 
\sup_{x,y \in \Omega} \frac{|f(x)-f(y)|}{|x-y|^\alpha} &= \sup |f(x)-f(y)|^{1-\alpha} \paren{\frac{|f(x)-f(y)|}{|x-y|}}^\alpha 
\\ &\leq \paren{2 \norm{f}_\infty}^{1-\alpha} \paren{ \sup \frac{|f(x)-f(y)|}{|x-y|} }^\alpha
\\ &\leq C \norm{f}_\infty^{1-\alpha} \norm{\grad f}_\infty^\alpha.
\end{align*}

The second claim is more complicated.  We will prove the stronger claim that for $f \in C^{1,\alpha}$
\[ \norm{\grad f}_\infty \leq C \delta\n \norm{f}_{L^\infty(\bar{\Omega})} + \delta^\alpha \bracket{\grad f}_{\alpha;\bar{\Omega}}.\]

The idea of the proof is to average $\grad f$ along an interval of length $\delta$ with endpoint $x$.  The magnitude of the average will be small, since $f \in L^\infty$, and the average will differ not very much from $\grad f(x)$ since $\grad f \in C^{1,\alpha}$.  

Since $\Omega$ satisfies the cone condition, there exist positive constants $\ell$ and $a<1$ such that, at each point $x \in \bar{\Omega}$, there exist two unit vectors $e_1$ and $e_2$ such that $|e_1\cdot e_2| \leq a$ and $x + \tau e_i \in \Omega$ for $i=1,2$ and $0 < \tau \leq \ell$.  In other words, $\Omega$ contains rays at each point that extend for length $\ell$, end at $x$, and are non-parallel with angle at least $\cos\n(a)$.  

Consider the directional derivative $\del_i f$ of $f$ along the direction $e_i$, and observe that for any $0 < \delta \leq \ell$,
\begin{equation} \label{average bounded by Linfty} \abs{\int_0^\delta \del_i f(x + \tau e_i) \,d\tau} = \abs{f(x+\delta e_i) - f(x)} \leq 2 \norm{f}_\infty. \end{equation}

On the other hand, $\del_i f$ is continous so, for any $\tau \in (0,\ell]$,
\[ \abs{\del_i f(x) - \del_i f(x+\tau e_i)} \leq \bracket{\grad f}_\alpha \tau^\alpha. \]
From this, we obtain that
\[ \int_0^\delta \del_i f(x + \tau e_i) \,d\tau \leq \int_0^\delta \paren{\del_i f(x) + \bracket{\grad f}_\alpha \tau^\alpha } \,d\tau = \delta \del_i f(x) + \bracket{\grad f}_\alpha \frac{\delta^{1+\alpha}}{1+\alpha} \]
and a similar bound holds from below.  Thus
\[ \abs{ \delta \del_i f(x) - \int_0^\delta \del_i f(x + \tau e_i) \,d\tau} \leq \bracket{\grad f}_\alpha \frac{\delta^{1+\alpha}}{1+\alpha}. \]

Combining this bound with \eqref{average bounded by Linfty}, we obtain
\[ \abs{\del_i f(x)} \leq \frac{2}{\delta} \norm{f}_\infty + \frac{\delta^\alpha}{1+\alpha} \bracket{\grad f}_\alpha. \]

This bound is independent of $x$ and of $i=1,2$.  Since $e_1 \cdot e_2 \leq a$ by assumption, by a little linear algebra we can bound $\grad f$ in terms of the $\del_i f$ and obtain that, for all $\delta \in (0,\ell]$,
\[ \norm{\grad f}_\infty \leq \frac{C}{1-a^2} \paren{ \delta\n \norm{f}_\infty + \delta^\alpha \bracket{\grad f}_\alpha }. \]

\end{proof}

%\section{Existence} \label{sec:existence}
%In \cite{CoIg.sqg}, appendix 2, it is argued that for initial data $\theta_0 \in H_0^1(\Omega) \cap H^2(\Omega)$, there exist local-in-time solutions to \eqref{eq:main nonlinear} with
%\[ \theta \in L^\infty(0, T; H_0^1(\Omega) ∩ H^2(\Omega)) \cap L^2(0,T; \HD^{2.5}). \]
%
%In particular, this is enough to show that $\theta \in L^\infty(0,T; L^4(\Omega))$ and hence that $u$ is as well.  Therefore the drift term $u \cdot \grad \theta$ is a product of an $L^2$ function and an $L^4$ function.  
%
%This is enough to justify all of the calculations in the present paper of the form
%\[ 0 = \int \varphi (\theta - \Psi)_+ \bracket{\del_t \theta + u \cdot \grad\theta + \Lambda \theta} \,dx \]
%for $\varphi$ an $L^\infty$ function and $\Psi$ a non-negative function.  
%
%Given $L^2(\Omega)$ initial data $\theta_0$, one can prove existence for a H\"{o}lder continuous solution on an interval $[0,T]$ by approximating $\theta_0$ by a sequence of sufficiently regular functions.  For this sequence, we apply the local existence theory and the H\"{o}lder theory to construct uniformly H\"{o}lder continuous solutions over the entire time interval $[0,T]$.  One can then use Arzela-Ascoli to construct a limit, which will be both a weak solution to \eqref{eq:main nonlinear} and a H\"{o}lder continuous function with slightly lower exponent.  

%-------%-------%-------%-------%-------%-------%-------%-------%-------%-------%-------%-------

%\section{Waxing Philosophical}
\bibliographystyle{alpha}
\bibliography{SQG-references}

\end{document}